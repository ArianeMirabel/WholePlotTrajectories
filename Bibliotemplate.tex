\documentclass[man]{apa6}

\usepackage{amssymb,amsmath}
\usepackage{ifxetex,ifluatex}
\usepackage{fixltx2e} % provides \textsubscript
\ifnum 0\ifxetex 1\fi\ifluatex 1\fi=0 % if pdftex
  \usepackage[T1]{fontenc}
  \usepackage[utf8]{inputenc}
\else % if luatex or xelatex
  \ifxetex
    \usepackage{mathspec}
    \usepackage{xltxtra,xunicode}
  \else
    \usepackage{fontspec}
  \fi
  \defaultfontfeatures{Mapping=tex-text,Scale=MatchLowercase}
  \newcommand{\euro}{€}
\fi
% use upquote if available, for straight quotes in verbatim environments
\IfFileExists{upquote.sty}{\usepackage{upquote}}{}
% use microtype if available
\IfFileExists{microtype.sty}{\usepackage{microtype}}{}

% Table formatting
\usepackage{longtable, booktabs}
\usepackage{lscape}
% \usepackage[counterclockwise]{rotating}   % Landscape page setup for large tables
\usepackage{multirow}		% Table styling
\usepackage{tabularx}		% Control Column width
\usepackage[flushleft]{threeparttable}	% Allows for three part tables with a specified notes section
\usepackage{threeparttablex}            % Lets threeparttable work with longtable

% Create new environments so endfloat can handle them
% \newenvironment{ltable}
%   {\begin{landscape}\begin{center}\begin{threeparttable}}
%   {\end{threeparttable}\end{center}\end{landscape}}

\newenvironment{lltable}
  {\begin{landscape}\begin{center}\begin{ThreePartTable}}
  {\end{ThreePartTable}\end{center}\end{landscape}}

  \usepackage{ifthen} % Only add declarations when endfloat package is loaded
  \ifthenelse{\equal{\string man}{\string man}}{%
   \DeclareDelayedFloatFlavor{ThreePartTable}{table} % Make endfloat play with longtable
   % \DeclareDelayedFloatFlavor{ltable}{table} % Make endfloat play with lscape
   \DeclareDelayedFloatFlavor{lltable}{table} % Make endfloat play with lscape & longtable
  }{}%



% The following enables adjusting longtable caption width to table width
% Solution found at http://golatex.de/longtable-mit-caption-so-breit-wie-die-tabelle-t15767.html
\makeatletter
\newcommand\LastLTentrywidth{1em}
\newlength\longtablewidth
\setlength{\longtablewidth}{1in}
\newcommand\getlongtablewidth{%
 \begingroup
  \ifcsname LT@\roman{LT@tables}\endcsname
  \global\longtablewidth=0pt
  \renewcommand\LT@entry[2]{\global\advance\longtablewidth by ##2\relax\gdef\LastLTentrywidth{##2}}%
  \@nameuse{LT@\roman{LT@tables}}%
  \fi
\endgroup}


\ifxetex
  \usepackage[setpagesize=false, % page size defined by xetex
              unicode=false, % unicode breaks when used with xetex
              xetex]{hyperref}
\else
  \usepackage[unicode=true]{hyperref}
\fi
\hypersetup{breaklinks=true,
            pdfauthor={},
            pdftitle={The title},
            colorlinks=true,
            citecolor=blue,
            urlcolor=blue,
            linkcolor=black,
            pdfborder={0 0 0}}
\urlstyle{same}  % don't use monospace font for urls

\setlength{\parindent}{0pt}
%\setlength{\parskip}{0pt plus 0pt minus 0pt}

\setlength{\emergencystretch}{3em}  % prevent overfull lines


% Manuscript styling
\captionsetup{font=singlespacing,justification=justified}
\usepackage{csquotes}
\usepackage{upgreek}

 % Line numbering
  \usepackage{lineno}
  \linenumbers


\usepackage{tikz} % Variable definition to generate author note

% fix for \tightlist problem in pandoc 1.14
\providecommand{\tightlist}{%
  \setlength{\itemsep}{0pt}\setlength{\parskip}{0pt}}

% Essential manuscript parts
  \title{The title}

  \shorttitle{Title}


  \author{First Author\textsuperscript{1}~\& Ernst-August Doelle\textsuperscript{1,2}}

  % \def\affdep{{"", ""}}%
  % \def\affcity{{"", ""}}%

  \affiliation{
    \vspace{0.5cm}
          \textsuperscript{1} Wilhelm-Wundt-University\\
          \textsuperscript{2} Konstanz Business School  }

  \authornote{
    Add complete departmental affiliations for each author here. Each new
    line herein must be indented, like this line.
    
    Enter author note here.
    
    Correspondence concerning this article should be addressed to First
    Author, Postal address. E-mail:
    \href{mailto:my@email.com}{\nolinkurl{my@email.com}}
  }


  \abstract{Enter abstract here. Each new line herein must be indented, like this
line.}
  \keywords{keywords \\

    \indent Word count: X
  }





\usepackage{amsthm}
\newtheorem{theorem}{Theorem}[section]
\newtheorem{lemma}{Lemma}[section]
\theoremstyle{definition}
\newtheorem{definition}{Definition}[section]
\newtheorem{corollary}{Corollary}[section]
\newtheorem{proposition}{Proposition}[section]
\theoremstyle{definition}
\newtheorem{example}{Example}[section]
\theoremstyle{definition}
\newtheorem{exercise}{Exercise}[section]
\theoremstyle{remark}
\newtheorem*{remark}{Remark}
\newtheorem*{solution}{Solution}
\begin{document}

\maketitle

\setcounter{secnumdepth}{0}

























\section{Introduction}\label{introduction}

The large areas covered with tropical forests worldwide hold crucial
environmental, economic and social values. They provide wood and
multiple non-timber forest products, shelter a diversified fauna,
regulate the local and regional climates, the carbon, water and nutrient
cycles, and ensure cultural and human well-being. The growing demand in
forests products together with current global changes increases the
pressure on remaining natural forests ({\textbf{???}}) and threatens the
maintenance and dynamics in space and time of communities structure,
composition and functioning ({\textbf{???}}).

In tropical forests, ecological communities are regularly re-shaped by
natural disturbance events changing both the abiotic environment,
through the fluxes of light, heat and water ({\textbf{???}}), and the
biotic interactions such as competition among species ({\textbf{???}}).
One of the cornerstone of tropical forest ecology is to understand the
processes and drivers of ecosystems response to disturbance
({\textbf{???}}). For now, this has been largely studied through forest
structural parameters such as aboveground biomass, tree height or stem
density ({\textbf{???}}, {\textbf{???}}) that are rapid and convenient
to measure. These structural parameters have been sucessfully modeled,
giving important insights into the recovery of ecosystem processes and
services ({\textbf{???}}). However the response of forests diversity and
composition remains unclear, albeit it determines the productivity,
stability and functioning of ecosystems ({\textbf{???}},
{\textbf{???}}). In the short-term, moderate disturbance may lead to
positive impacts on communities diversity, an idea formalized by the
intermediate disturbance hypothesis (IDH) stating a maximized species
diversity when disturbance intensity is not too high ({\textbf{???}},
{\textbf{???}}).

Validations of the IDH though remain scarce in the long-term and mainly
rely on the analysis of taxonomic richness ({\textbf{???}}). Taxonomic
richness alone, however, gives limited or misleading information on
forests recovery and functioning ({\textbf{???}}). More
ecological-meaningful analysis would couple richness with (i) evenness,
that would reveal the changes in the species abundance distribution and
thus the underlying ecological processes, and (ii) composition that is
crucial for conservation issues ({\textbf{???}}, {\textbf{???}}).
Furthermore, a functional approach accounting for species biological
attributes would directly link communities diversity, composition and
redundancy to ecosystem functioning and to its environmental constraints
({\textbf{???}}, {\textbf{???}}). In that respect, the functional
trait-based approach that focus on major traits related to species
ecology and mediate species performance in a given environment was
sucessfully adopted ({\textbf{???}}). For instance, the functional
approach revealed in tropical rainforests the deterministic processes
entailing, after disturbance, a functional shift from a dominance of
\enquote{conservative} slow-growing species dealing with scarce
resources to \enquote{acquisitive} fast-growing species with rapid and
efficient use of abundant resources ({\textbf{???}}, {\textbf{???}}).
This shift is translated into the trajectories of key functional traits
related to resource acquisition (leaf and stem traits) and life-history
traits (seed mass, maximum size) ({\textbf{???}}, {\textbf{???}},
{\textbf{???}}, {\textbf{???}}). Eventually a complete overview of
communities response to disturbance would encompass the changes in
functional redundancy, that quantifies the amount of shared trait values
among species ({\textbf{???}}). The high functional redundancy of
hyperdiverse tropical forests ({\textbf{???}}) mitigates the impacts of
species removal on ecosystem functioning and determines the resilience
of communities after disturbance ({\textbf{???}}, {\textbf{???}}).

In this study, we monitored over 30 years the response of 75 ha of
neotropical forest plots set up on a gradient of disturbance intensity,
from 10 to 60\% of ecosystem above-ground biomass (AGB) loss. We made
use of a large functional traits database encompassing major leaf, stem
and life-history traits in order to draw the taxonomic and functional
trajectories in terms of richness, evenness, composition and redundancy.
Specifically, we (i) elucidated community taxonomic and functional
recovery and the underlying ecological processes, (ii) clarified the
validity of the IDH in the long term for tropical forest and its
translation into different trajectories in time, and (iii) questioned
community recovery time.

\section{Material and Methods}\label{material-and-methods}

\subsection{Study site}\label{study-site}

Paracou station in French Guiana (5\textdegree 18'N and
52\textdegree 53'W) is located in a lowland tropical rain forest in a
tropical wet climate with mean annual temperature of 26\textdegree C,
mean annual precipitation averaging 2980 mm.y\textsuperscript{-1} (30-y
period) and a 3-month dry season (\textless{} 100
mm.month\textsuperscript{-1}) from mid-August to mid-November, and a
one-month dry season in March ({\textbf{???}}). Elevation ranges between
5 and 50 m and soils correspond to thin acrisols over a layer of
transformed saprolite with low permeability generating lateral drainage
during heavy rains.

The experiment is a network of twelve 6.25ha plots that underwent a
disturbance gradient of three logging, thinning and fuelwood cutting
treatments (Table \ref{tab:Tab1}) according to a randomized plot design
with three replicate blocks of four plots. The disturbance corresponds
to averages of 10 trees removed per hectare with a diameter at 1.3 m
height (DBH) above 50 cm for treatment 1 (T1), 32 trees/ha above 40 cm
DBH for treatment 2 (T2) and 40 trees/ha above 40 cm DBH for treatment 3
(T3). Treatments T2 and T3 besides included the thinning of trees by
poison girdling ({\textbf{???}}). The disturbance intensity was measured
as the percentage of aboveground biomass (\%AGB) lost between the first
inventory in 1984 and five years after disturbance ({\textbf{???}})
estimated with the BIOMASS R package ({\textbf{???}}).

\subsection{Inventories protocol and dataset
collection}\label{inventories-protocol-and-dataset-collection}

The study site corresponds to a tropical rainforest typical of the
Guiana Shield with a dominance of Fabaceae, Chrysobalanaceae,
Lecythidaceae and Sapotaceae. In the twelve experimental plots, all
trees above 10 cm DBH have been mapped and measured annually since 1984.
Trees are first identified with a vernacular name assigned by the forest
worker team, and afterward with a scientific name assigned by botanists
during regular botanical campaigns. In 1984, specific vernacular names
were given to 62 commercial or common species whereas more infrequent
ones were identified under general identifiers only distinguishing trees
and palm trees. From 2003, botanical campaigns have been conducted every
5 to 6 years to identify all trees at the species level but
identification levels still varied among plots and campaigns.

This variability of protocols in time raised methodological issues as
vernacular names usually correspond to different botanical species. It
resulted in significant taxonomic uncertainties that had to be
propagated to composition and diversity metrics. The uncertainty
propagation was done through a Bayesian framework reconstituting
complete inventories at genus level from real incomplete ones on the
basis of vernacular/botanical names association. Vernacular names were
replaced through multinomial trials based on the association probability
\(\big[\alpha_1, \alpha_2,..., \alpha_V\big]\) observed across all
inventories between each vernacular name \emph{v} and all species
\(\big[s_1, s_2,..., s_N\big]\):

\begin{align}
M_v\Big(\big[s_1, s_2,..., s_N\big],\big[\alpha_1, \alpha_2,..., \alpha_V\big]\Big) \nonumber
\end{align}

See Supplementary Materials -Figure S1 and ({\textbf{???}}) for the
detailed methodology.

Six functional traits representing leaf economics (leaves thickness,
toughness, total chlorophyll content and specific leaf area) and stem
economics (wood specific gravity and bark thickness), and life-history
traits (maximum specific height and seed mass) came from the BRIDGE
project.\footnote{\url{http://www.ecofog.gf/Bridge/}} Trait values were
assessed from a selection of individuals located in nine permanent plots
in French Guiana, including two in Paracou, and comprised 294 species
pertaining to 157 genera. Missing trait values (10\%) were filled using
multivariate imputation by chained equation ({\textbf{???}}).
Imputations were restricted within genus or family when samples were too
scarce, in order to account for the phylogenetic signal. Whenever a
species was not in the dataset, it was attributed a set of trait values
randomly sampled among species of the next higher taxonomic level (same
genus or family). As seed mass information was classified into classes,
no data filling process was applied and analyses were restricted to the
414 botanical species recorded.

All composition and diversity metrics were obtained after 50 iterations
of the uncertainty propagation framework.

\subsection{Composition and diversity
metrics}\label{composition-and-diversity-metrics}

To counter taxonomic uncertainties due to the variability of botanical
identification levels (in space) and protocols (in time), the taxonomic
composition and diversity analysis were conducted at the genus level.
Taxonomic and functional trajectories of community composition were
followed in a two-dimensional NMDS ordination space. Two NMDS using
abundance-based (Bray-Curtis) dissimilarity measures were conducted to
map either taxonomic or functional composition, the later based on the 7
leaf, stem and life history traits (without seed mass classes).
Trajectories along time were reported through the euclidean distance
between the target inventories and the reference inventories in 1989,
\emph{i.e.} 2 years after disturbance. Univariate trajectories of the
leaf, stem and life-history traits were also visualized with the
community weighted means (CWM) ({\textbf{???}}). Species seed mass
corresponded to 5 classes of increasing mass, seed mass trajectories
were therefore reported as the proportion of each class in the
inventories (Supplementary materials).

The taxonomic diversity was reported through species richness and
evenness, \emph{i.e} the Hill number translation of the Simpson index
({\textbf{???}}). These indices belong to the set of HCDT or generalized
entropy, respectively corresponding to the 0 and 2 order of diversity
(q), recomended for diversity studies ({\textbf{???}}). The functional
diversity was reported using the functional richness and functional
evenness, \emph{i.e} Rao index of quadratic entropy which combines
species abundance distribution and average pairwise dissimilarity based
on species functional traits.

The impacts of initial disturbance were tested with the spearman rank
correlation between the extrema of taxonomic and functional metrics
reached over the 30 years and the initial \%AGB loss. They were besides
analysed through polynomial regression between (i) taxonomic and
functional richness and evenness and (ii) the initial \%AGB loss at 10,
20 and 30 years after disturbance.

The functional redundancy was measured as the overlap among species in
community functional space ({\textbf{???}}). The samples of the trait
database were first mapped in a 2-dimensional plan with a PCA analysis.
Then, multivariate kernel density estimator associated with individual
trees returned species traits probability distribution (TDP). Species
TDP weighted by species abundance were eventually summed for each
community. Community functional redundancy was the sum of TDPs overlap,
expressed as the average number of species that could be removed from
without reducing the functional space (Supp. Mat. - Figure S1 for a more
comprehensive sheme).

\section{Results}\label{results}

\subsection{Communities Composition}\label{communities-composition}

From 1989 (2 years after disturbance) to 2015 (28 years after
disturbance), 828 388 individual trees and 591 botanical species
pertaining to 223 genus and 64 families were recorded.

While both taxonomic and functional composition remained stable in
undisturbed communities (Figure \ref{fig:NMDSplans}), they followed
marked and consistent trajectories over post-\break disturbance time. In
disturbed communities, these compositional changes corresponded to
shifts towards species with more acquisitive functional strategies, from
communities with high average WSG to high average SLA and chlorophyll
content (see appendix I). For functional composition, this translated
into cyclic compositional changes with an unachieved recovery of the
initial composition (Figure \ref{fig:NMDSplans}). The maximum
dissimilarity with the initial state was positively correlated to the
disturbance intensity for both taxonomic and functional composition
(\(\rho_{spearman}^{taxonomic}=0.87\) and
\(\rho_{spearman}^{functional}=0.90\) respectively). The maximum value
was reached around 26 years after disturbance for taxonomic composition
and 22 years for functional composition.

Except for leaf chlorophyll content, which continued to increase for
some T3 and T2 plots 30 years after disturbance, all traits and seed
mass proportions followed unimodal trajectories either stabilizing or
returning towards their initial values.

Maximum height at adult stage (\emph{Hmax}), leaf toughness
(\emph{L\_toughness}) and wood specific gravity (\emph{WSG}) first
decreased and then slightly increased but remained significantly lower
than their initial value (Figure \ref{fig:CWM}). On the other side, Bark
thickness (\emph{Bark\_thick}) and specific leaf area (\emph{SLA})
increased and while \emph{Bark\_thick} remained substantially high after
30 years, \emph{SLA} had almost recovered its initial value. For all
traits, the maximum difference to initial value was correlated to the
disturbance intensity (\(\rho_{spearman}^{L_{thickness}}=0.76\),
\(\rho_{spearman}^{L_{chloro}}=0.60\),
\(\rho_{spearman}^{L_{toughness}}=-0.53\),
\(\rho_{spearman}^{SLA}=0.93\), \(\rho_{spearman}^{WSG}=-0.75\),
\(\rho_{spearman}^{Bark-thickness}=0.71\),
\(\rho_{spearman}^{Hmax}=-0.40\)). The proportions of the three lightest
seed mass classes increased in all disturbed plots, and decreased after
30 years for the lightest class while it stabilized for the two other
(Supp. Mat. - Figure S2).

\subsection{Communities richness and
evenness}\label{communities-richness-and-evenness}

For undisturbed plots, taxonomic Richness and Evenness remained stable
over the 30 years monitored. In disturbed communities, after low
disturbance intensity the taxonomic richness increased, reaching a
maximum gain of 14 botanical genera (plot 3 from treatment 2). After
intense disturbance the taxonomic richness followed a more complex
trajectory, decreasing for ten years after disturbance before recovering
to pre-disturbance values. The maximum richness loss or gain after
disturbance was positively correlated to the disturbance intensity
(\(\rho_{spearman}^{Richness}=0.50\)). In all disturbed plots the
taxonomic evenness first increased until a maximum reached after around
20 years. This maximum was positively correlated to the disturbance
intensity (\(\rho_{spearman}^{Evenness}=0.77\)). The evenness then
stabilized except for two T3 plots (plots 8 and 12) for which evenness
kept increasing.

The plot 7 from treatment 1 displayed constantly outlying functional
richness and evenness and was removed from the graphical representation
for better readability. In undisturbed plots both functional richness
and evenness remained stable along the 30 years. In disturbed plots,
functional richness and evenness trajectories depended on the
disturbance intensity with their maximum positively correlated to \%AGB
loss \(\rho_{spearman}^{Richness}=0.76\) and
\(\rho_{spearman}^{Evenness}=0.60\). Functional richness and evenness
displayed for low disturbance intensity a low but long-lasting increase
up to a maximum reached after 20-25 years, and for high intensity, a
fast but short increase followed after 10 years by a slow decrease
towards the inital values.

The second-degree polynomial regressions between (i) the \%AGB loss and
(ii) taxonomic and functional richness and evenness after 10, 20 and 30
years best predicted the hump-shaped curve of the disturbance impact
along the disturbance intensity gradient \ref{fig:IDHplot}. The
relationship between the disturbance impact and its intensity was more
markedly hump-shaped for the taxonomic richness than for the taxonomic
evenness. For both functional richness and evenness the relationship was
almost linear. The regression model better predicted the functional
richness and evenness (\(0.55<R^2_{Functional Richness}<0.72\), and
\(0.60<R^2_{Functional Evenness}<0.81\)) than the taxonomic richness and
evenness (\(0.21<R^2_{Taxonomic Richness}<0.4\), and
\(-0.15<R^2_{Taxonomic Evenness}<0.43\) respectively)

\subsection{Functional redundancy}\label{functional-redundancy}

All disturbed plots had lower functional redundancy than control plots
and followed similar hump-shaped trajectories (\ref{fig:RedFunRest}).
The maximum redundancy loss was positively correlated with the
disturbance intensity (\(\rho_{spearman}=0.47\)) and the initial value
had not recovered for any disturbed communities after 30 years.

\section{Discussion}\label{discussion}

\subsection{A cyclic recovery of community
composition}\label{a-cyclic-recovery-of-community-composition}

Communities taxonomic and functional composition appeared resilient,
following similar hump-shaped trajectories starting to return towards
pre-disturbance composition after 30 years.

The taxonomic differences among local communities, marked before
disturbance by the distinct starting points on the NMDS axis 2, were
maintained throughout recovery trajectories. More than commonly thought,
post-disturbance trajectories depended on community initial composition,
that partly determined the pool of recruited species and constrained the
trajectories towards the initial composition. The high resilience of
communities taxonomy revealed that species not belonging to the
pre-disturbance community were hardly recruited because of the
commonness of dispersal limitation among tropical tree species
({\textbf{???}}).

Conversely, disturbed communities followed functional trajectories that
are highly similar in terms of functional composition. As
pre-disturbance surviving trees mirror the initial community
({\textbf{???}}), changes in functional composition relied upon the
recruitment of species or functional types that were infrequent or
absent before disturbance. Competitive pioneers became dominant in
filling the environmental niches of high availability of light, space
and nutrients vacated by the disturbance. The recruitment of pioneers
changed community functional composition in the same way for all
disturbance intensity towards more resource-acquisitive strategies,
moving community functional composition right along the first axis in
Figure \ref{fig:NMDSplans} ({\textbf{???}}, {\textbf{???}},
{\textbf{???}}). Thereafter long-lived, more resistant and
shade-tolerant species excluded the first established pioneers and
started the recovery of pre-disturbance functional composition, moving
similarly community functional composition left along the first axis and
upward along the second axis in Figure \ref{fig:NMDSplans}.

These trajectories provided empirical support to the hypothesis that
community assembly is both deterministic and historically convergent at
different levels of community organization. Deterministic, trait-based
processes drove community convergence in functional composition, while
at the same time dispersal limitation maintained their divergence in
taxonomic composition ({\textbf{???}}).

\subsection{Another perspective on the intermediate disturbance
hypothesis}\label{another-perspective-on-the-intermediate-disturbance-hypothesis}

The IDH well predicted well the disturbance impact on community
taxonomic richness, enhanced until an intensity threshold (20-25\% AGB
loss), and to some extent on taxonomic evenness, somewhat decoupled from
the disturbance intensity as already observed in the Guiana Shield
({\textbf{???}}) and in Bornean tropical forests ({\textbf{???}}). The
disturbance intensity determined the balance in the community between
pre-disturbance surviving trees and those recruited afterward. The pool
of true pioneer species specifically recruited after disturbance is
restricted in the Guiana Shield to a few common genera (e.g.~Cecropia
spp., Vismia spp.) ({\textbf{???}}). Below the intensity threshold the
size of the surviving community maintained the pre-disturbance high
taxonomic richness while the recruitment of pioneers, infrequent or
absent before disturbance, increased both community taxonomic richness
and evenness. Beyond the intensity threshold, the disturbance decreased
the taxonomic richness of surviving trees which was not offset by the
enrichment of pioneers, so that the overall community taxonomic richness
decreased according to the disturbance intensity ({\textbf{???}}). For
community taxonomic evenness the disturbance impact was similar but
slighter, as the evenness is less sensitive to the loss of rare species.
Taxonomic evenness rather represented the increasing dominance of
pionneers that balanced the usual hyper-dominance of a few species in
tropical forests below the intensity threshold, thus increasing
community overall evenness up to the intensity threshold beyond which
pioneers became in turn highly dominant and decreased the overall
evenness ({\textbf{???}}).

Conversely the IDH was disproved regarding the disturbance impact on
community functional richness and evenness. Irrespective of the
disturbance intensity the recruitment of pionneers, functionally highly
different from the composition of pre-disturbance community, increased
both community functional richness and eveness.

Along time, taxonomic richness trajectories of all disturbed communities
first dropped similarly, following the species loss due to disturbance,
and then displayed a species gain depending on the disturbance
intensity. Up to an intensity threshold, the species gain was all the
more significant that the disturbance intensity increased, with the
establishment of long-lived pioneers enhancing community taxonomic
richness and evenness in the long term. These long-lived pionneers,
functionally quite different from the functional composition, entailed
as well a progressive and long-lasting increase of the functional
richness and evenness ({\textbf{???}}, {\textbf{???}}). Beyond an
intensity threshold, though, a few short-lived pioneers occupied the
vacated environmental space and prevented the establishment of other
species. These short-lived pioneers were functionally very different
from the pre-disturbance community and entailed a rapid and significant
increase of functional richness en evenness. Already after 10 years,
though, short-lived pioneers started to decline and the functional
richness and evenness decreased. Likely this decrease will be followed
by the establishment of long-lasting pioneers, and by the time they
recruit we expect the taxonomic and functional trajectories to catch up
with those observed after intermediate disturbance ({\textbf{???}}).

\subsection{The functional redundancy, key of community
resilience}\label{the-functional-redundancy-key-of-community-resilience}

For 15 years the species loss during disturbance, determined by the
disturbance intensity, commensurately decreased the functional
redundancy within the pre-disturbance functional space. The redundancy
decrease was not compensated in the first place as the first recruited
pionneers were functionally different from the pre-disturbance
functional composition. Progressively though, first established species
were replaced by more competitive long-lived pionneers or
late-successional species resembling more the pre-disturbance functional
composition and restoring the functional redundancy. This replacement
was stochastic and followed the lottery recruitment rules, implying a
recruitment eased for the first recruited species but then increasingly
hampered by the emergence of interspecific competition ({\textbf{???}}).
Along time the recovery of infrequent species was increasingly slow, so
that the time for the full recovery of the functional redundancy, in
some communities just initiated after 30 years, was extremely difficult
to estimate ({\textbf{???}}, {\textbf{???}}).

The long-term impact of disturbance on community functional redundancy
meant a lower resilience of the pre-disturbance communities, with higher
chances to see the persistence of disturbance-specific species at the
expense of late-successional ones ({\textbf{???}}). Besides, the
long-term recovery of infrequent species increases the risks to loose
cornerstone species, with unexpected ecological consequences
({\textbf{???}}, {\textbf{???}}, {\textbf{???}}). Apart from the
functional characteristics considered here, infrequent species might
indeed have unique functions in the ecosystem or be a key for some fauna
({\textbf{???}}).

\section{Conclusions}\label{conclusions}

Our study revealed community recovery through the combination of
deterministic processes driving their convergence in functional
composition, and dispersal limitation maintaining their divergence in
taxonomic composition. The IDH was validated for community taxonomic
richness and, to some extent, taxonomic evenness but disproved regarding
community functional richness and evennes that were enhanced for any
disturbance intensity by the high functional differences of pioneers
compared to late-successional functoinal composition. The IDH was
translated in time by the recruitment, beyond an intensity threshold, of
short-lived pionneers that prevented in the first times the
establishment of more diverse long-lived pionneers, recruited otherwise
below the intensity threshold. The resilience of tropical forests proved
consistent although several decades-long. Still, the disturbance impact
on communities redundancy cautioned against the risks of infrequent
species loss and the persistence of disturbance-specific communities
({\textbf{???}}).

\newpage

\section{References}\label{references}

\begingroup
\setlength{\parindent}{-0.5in} \setlength{\leftskip}{0.5in}

\hypertarget{refs}{}

\endgroup






\end{document}
