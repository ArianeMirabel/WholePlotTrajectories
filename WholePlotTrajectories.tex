%%%%%%%%%%%%%%%%%%%%%%%%%%%%%%%%%%%%%%%%%
% Article EcoFoG
% Version 2.1 (23/10/2017)
%
% adapté de :
% Stylish Article
% LaTeX Template
% Version 1.0 (31/1/13)
%
% This template has been downloaded from:
% http://www.LaTeXTemplates.com
%
% Original author:
% Mathias Legrand (legrand.mathias@gmail.com)
%
% License:
% CC BY-NC-SA 3.0 (http://creativecommons.org/licenses/by-nc-sa/3.0/)
%
%%%%%%%%%%%%%%%%%%%%%%%%%%%%%%%%%%%%%%%%%


%----------------------------------------------------------------------------------------
%	PACKAGES AND OTHER DOCUMENT CONFIGURATIONS
%----------------------------------------------------------------------------------------

\documentclass[fleqn,10pt]{ArtEcoFoG} % Document font size and equations flushed left

\setcounter{tocdepth}{3} % Show only three levels in the table of contents section: sections, subsections and subsubsections


% Pandoc environments
\usepackage{framed}
\usepackage{fancyvrb}
\providecommand{\tightlist}{%
  \setlength{\itemsep}{0pt}\setlength{\parskip}{0pt}}
\newcommand{\VerbBar}{|}
\newcommand{\VERB}{\Verb[commandchars=\\\{\}]}
\DefineVerbatimEnvironment{Highlighting}{Verbatim}{commandchars=\\\{\}, fontsize=\scriptsize} % Code R
\definecolor{shadecolor}{RGB}{248,248,248}
\newenvironment{Shaded}{\begin{snugshade}}{\end{snugshade}}
\newcommand{\KeywordTok}[1]{\textcolor[rgb]{0.13,0.29,0.53}{\textbf{{#1}}}}
\newcommand{\DataTypeTok}[1]{\textcolor[rgb]{0.13,0.29,0.53}{{#1}}}
\newcommand{\DecValTok}[1]{\textcolor[rgb]{0.00,0.00,0.81}{{#1}}}
\newcommand{\BaseNTok}[1]{\textcolor[rgb]{0.00,0.00,0.81}{{#1}}}
\newcommand{\FloatTok}[1]{\textcolor[rgb]{0.00,0.00,0.81}{{#1}}}
\newcommand{\ConstantTok}[1]{\textcolor[rgb]{0.00,0.00,0.00}{{#1}}}
\newcommand{\CharTok}[1]{\textcolor[rgb]{0.31,0.60,0.02}{{#1}}}
\newcommand{\SpecialCharTok}[1]{\textcolor[rgb]{0.00,0.00,0.00}{{#1}}}
\newcommand{\StringTok}[1]{\textcolor[rgb]{0.31,0.60,0.02}{{#1}}}
\newcommand{\VerbatimStringTok}[1]{\textcolor[rgb]{0.31,0.60,0.02}{{#1}}}
\newcommand{\SpecialStringTok}[1]{\textcolor[rgb]{0.31,0.60,0.02}{{#1}}}
\newcommand{\ImportTok}[1]{{#1}}
\newcommand{\CommentTok}[1]{\textcolor[rgb]{0.56,0.35,0.01}{\textit{{#1}}}}
\newcommand{\DocumentationTok}[1]{\textcolor[rgb]{0.56,0.35,0.01}{\textbf{\textit{{#1}}}}}
\newcommand{\AnnotationTok}[1]{\textcolor[rgb]{0.56,0.35,0.01}{\textbf{\textit{{#1}}}}}
\newcommand{\CommentVarTok}[1]{\textcolor[rgb]{0.56,0.35,0.01}{\textbf{\textit{{#1}}}}}
\newcommand{\OtherTok}[1]{\textcolor[rgb]{0.56,0.35,0.01}{{#1}}}
\newcommand{\FunctionTok}[1]{\textcolor[rgb]{0.00,0.00,0.00}{{#1}}}
\newcommand{\VariableTok}[1]{\textcolor[rgb]{0.00,0.00,0.00}{{#1}}}
\newcommand{\ControlFlowTok}[1]{\textcolor[rgb]{0.13,0.29,0.53}{\textbf{{#1}}}}
\newcommand{\OperatorTok}[1]{\textcolor[rgb]{0.81,0.36,0.00}{\textbf{{#1}}}}
\newcommand{\BuiltInTok}[1]{{#1}}
\newcommand{\ExtensionTok}[1]{{#1}}
\newcommand{\PreprocessorTok}[1]{\textcolor[rgb]{0.56,0.35,0.01}{\textit{{#1}}}}
\newcommand{\AttributeTok}[1]{\textcolor[rgb]{0.77,0.63,0.00}{{#1}}}
\newcommand{\RegionMarkerTok}[1]{{#1}}
\newcommand{\InformationTok}[1]{\textcolor[rgb]{0.56,0.35,0.01}{\textbf{\textit{{#1}}}}}
\newcommand{\WarningTok}[1]{\textcolor[rgb]{0.56,0.35,0.01}{\textbf{\textit{{#1}}}}}
\newcommand{\AlertTok}[1]{\textcolor[rgb]{0.94,0.16,0.16}{{#1}}}
\newcommand{\ErrorTok}[1]{\textcolor[rgb]{0.64,0.00,0.00}{\textbf{{#1}}}}
\newcommand{\NormalTok}[1]{{#1}}
\usepackage{longtable,booktabs}
\usepackage{caption}
% These lines are needed to make table captions work with longtable:
\makeatletter
\def\fnum@table{\tablename~\thetable}
\makeatother
% longtable 2 columns
% https://tex.stackexchange.com/questions/161431/how-to-solve-longtable-is-not-in-1-column-mode-error
\makeatletter
\let\oldlt\longtable
\let\endoldlt\endlongtable
\def\longtable{\@ifnextchar[\longtable@i \longtable@ii}
\def\longtable@i[#1]{\begin{figure}[t]
\onecolumn
\begin{minipage}{0.5\textwidth}\scriptsize
\oldlt[#1]
}
\def\longtable@ii{\begin{figure}[t]
\onecolumn
\begin{minipage}{0.5\textwidth}\scriptsize
\oldlt
}
\def\endlongtable{\endoldlt
\end{minipage}
\twocolumn
\end{figure}}
\makeatother

\usepackage{graphicx,grffile}
\makeatletter
\def\maxwidth{\ifdim\Gin@nat@width>\linewidth\linewidth\else\Gin@nat@width\fi}
\def\maxheight{\ifdim\Gin@nat@height>\textheight0.8\textheight\else\Gin@nat@height\fi}
\makeatother
% Scale images if necessary, so that they will not overflow the page
% margins by default, and it is still possible to overwrite the defaults
% using explicit options in \includegraphics[width, height, ...]{}
\setkeys{Gin}{width=\maxwidth,height=\maxheight,keepaspectratio}

% User-adder preamble
\usepackage{textcomp} \DeclareUnicodeCharacter{B0}{\textdegree}
\usepackage{tabu} \usepackage{caption}
\captionsetup{justification = justified}
\renewenvironment{table}{\begin{table*}}{\end{table*}\ignorespacesafterend}
\hyphenation{bio-di-ver-si-ty sap-lings}

%----------------------------------------------------------------------------------------
%	ARTICLE INFORMATION
%----------------------------------------------------------------------------------------

\JournalInfo{Hal xxx} % Journal information
\Archive{DOI xxxx} % Additional notes (e.g. copyright, DOI, review/research article)

\PaperTitle{Post-Disturbance Tree Community Trajectories in a Neotropical Forest} % Article title

\Authors{
Ariane MIRABEL\textsuperscript{1*}\\ Eric Marcon\textsuperscript{1}\\ Bruno Hérault\textsuperscript{2 3}
} % Authors
\affiliation{
\textsuperscript{1}UMR EcoFoG, AgroParistech, CNRS, Cirad, INRA, Université des Antilles,
Université de Guyane.\\ \hspace{1em} Campus Agronomique, 97310 Kourou, France.\\\textsuperscript{2}Cirad, Univ montpellier, UR Forests \& Societies.\\ \hspace{1em} Montpellier, France.\\\textsuperscript{3}INPHB, Institut National Polytechnique Félix Houphouet-Boigny\\ \hspace{1em} Yamoussoukro, Ivory Coast.
}
\affiliation{*\textbf{Corresponding author}: ariane.mirabel@ecofog.gf, http://www.ecofog.gf/spip.php?article47} % Corresponding author

\Keywords{mot-clés, séparés par des virgules} % Keywords - if you don't want any simply remove all the text between the curly brackets
\newcommand{\keywordname}{Keywords} % Defines the keywords heading name

%----------------------------------------------------------------------------------------
%	ABSTRACT
%----------------------------------------------------------------------------------------

\Abstract{
Résumé de l'article.
}

%----------------------------------------------------------------------------------------

\usepackage{amsthm}
\newtheorem{theorem}{Theorem}[section]
\newtheorem{lemma}{Lemma}[section]
\theoremstyle{definition}
\newtheorem{definition}{Definition}[section]
\newtheorem{corollary}{Corollary}[section]
\newtheorem{proposition}{Proposition}[section]
\theoremstyle{definition}
\newtheorem{example}{Example}[section]
\theoremstyle{definition}
\newtheorem{exercise}{Exercise}[section]
\theoremstyle{remark}
\newtheorem*{remark}{Remark}
\newtheorem*{solution}{Solution}
\begin{document}

\selectlanguage{english}

\flushbottom % Makes all text pages the same height

\maketitle % Print the title and abstract box

\tableofcontents % Print the contents section

\thispagestyle{empty} % Removes page numbering from the first page

%----------------------------------------------------------------------------------------
%	ARTICLE CONTENTS
%----------------------------------------------------------------------------------------
























\section{Introduction}\label{introduction}

The large areas covered with tropical forests worldwide hold crucial
economic, social and cultural values. They provide wood and multiple
non-timber forest products, shelter a diversified fauna, regulate the
local climate, support the carbon, water and nutrient cycles, and ensure
cultural and human well-being. The simultaneous increase of forests
products demand and the substantial climatic changes currently heighten
the pressure on remaining forests
\citep{Gibson2011a, Morales-Hidalgo2015}, threatening the maintenance of
communities structure, composition and functioning and their dynamics in
space and time \citep{Anderson-Teixeira2013, Sist2015}.

In tropical forest, ecological communities are constantly re-shaped by
the natural disturbance events changing both abiotic environment,
through the fluxes of light, heat and water, and biotic interactions and
competitive pressure \citep{Goulamoussene2017}. The cornerstone of
tropical forests ecology is to understand the mechanisms and the
determinants of ecosystems response to disturbance
\citep{White2001, Chazdon2003a}. For now, this has been largely studied
through structural parameters, rapid and convenient to measure, as
aboveground biomass, tree height or stem density
\citep{Piponiot2016, Rutishauser2016}. These structural parameters were
thereafter consistently modeled and gave important insights into the
maintenance of ecosystems processes and services
\citep{Denslow2000, Blanc2009}. However the response of forests
diversity in tree species remains unclear, albeit it determines the
productivity, stability and functioning of ecosystems
\citep[\citet{Liang2016}]{Tilman2014} and would be most probably
impacted by the changes following disturbance \citep{Baraloto2012a}.

In the short-term disturbance demonstrated negligible or even positive
impacts on communities diversity, which have been formalized by the
intermediate disturbance hypothesis (IDH) stating a maximized species
diversity at intermediate disturbance intensity
\citep{Molino2001, Kariuki2006a, Berry2008a}. Still, validations of the
IDH remain scarce in the long term and mainly rely on species richness
analyses that gives limited or misleading information on forests
recovery and functioning \citep{Martin2015, Chaudhary2016}. More
releveant monitoring would encompass communities composition, crucial
for conservation issues, and complete diversity profiles encompassing
species evenness in addition to their richness, that reveals ecological
rules shaping communities
\citep{Magurran1988, Lavorel2002, Bellwood2006}. Furthermore, account
for species biological attributes and role in the ecosystem is essential
to understand the correlations between ecosystems biodiversity,
functioning and environmental constraints
\citep{Violle2007b, Moretti2009, Baraloto2012a, Scheiter2013}. In that
respect functional approaches based on major traits related to species
ecology and performance was largely adopted
\citep{Diaz2005, Villeger2008a}, for example highlighting in tropical
rainforests the environmental filters fostering fast growing species
with efficient resources acquisition after disturbance
\citep{Molino2001, Haddad2008}. Post-disturbance trajectories then
corresponded to a shift from ``conservative'' slow-growing species
dealing with scarce resources dominating before disturbance, to
``acquisitive'' fast-growing species with rapid and efficient use of
abundant resources \citep{TerSteege2001, Reich2014, Herault2011}. This
shift translated into consistent trajectories for key functional traits
related to resource acquisition (leaf area, density and chlorophyll
content, and stem specific gravity and bark thickness), tree growth and
reproduction and life history traits (seed mass and maximum height)
\citep{Wright2004, Westoby2006a, Chave2009b}.

Functional and taxonomic trajectories are both essentiall to fully
assess communities response to disturbance. Both relate to distinct
ecological rules and processes and might then be decoupled as it was
observed already in tropical forests
\citep{Lohbeck2015, Guariguata2001}. Although taxonomic and functional
diversity are complementary, they are combine into communities
functional redundancy that measures the amount of species sharing same
trait values \citep{Carmona2016}. Functional redundancy then explicits
the link between otherwise decoupled taxonomic and functional
diversities and is central for communities description. Besides,
functional redundancy determines communities resilience as high
redundancy, like in highly diverse tropical forests
\citep{Bellwood2006}, mitigates the impacts of species removal on
ecosystem functioning \citep{Trenbath1999, Elmqvist2003, Diaz2005}.

Grasp all facets of communities response to disturbance amount to
disentangle the taxonomic and functional trajectories in diversity and
composition. These trajectories would highlight the ecological rules
constraining, or not, communities dynamics towards a recovery of initial
composition, diversity and functioning. They would therefore clarify the
tenants in the long term of the debated Intermediate Disturbance
Hypothesis for tropical forests, which would be as much insights for
future adaptive conservation strategies \citep{Adler2007}. Here we
monitored over 30 years the response of 75 ha of forests plots set up on
a gradient of disturbance intensity, from 10 to 60\% of ecosystem
biomass removed. We made use of a large functional traits database
browsing major leaf, stem and seed traits and species maximum height to
draw the trajectories over time of communities taxonomic and functional
composition and diversity. Specifically, we (i) questioned the coupling
between taxonomic and functional response to disturbance and identified
the underlying assembly processes, which allowed to (ii) clarify the
validity of the IDH in the long term for tropical forest and (iii)
question the completeness of communities recovery regarding their
functional redundancy.

\section{Material and Methods}\label{material-and-methods}

\subsection{Study site}\label{study-site}

Paracou station in French Guiana (5°18'N and 52°53'W) is located in a
lowland tropical rain forest in a tropical wet climate with mean annual
temperature of 26°C, mean annual precipitation averaging 2980
mm.y\textsuperscript{-1} (30-y period) and a 3-month dry season
(\textless{} 100 mm.month\textsuperscript{-1}) from mid-August to
mid-November, and a one-month dry season in March \citep{Wagner2011}.
Elevation ranges between 5 and 50 m and soils correspond to thin
acrisols over a layer of transformed saprolite with low permeability
generating lateral drainage during heavy rains.

The experiment is a network of twelve 6.25ha plots that underwent a
gradient of three logging, thinning and fuelwood cutting treatments
(Table \ref{tab:Tab1}). Disturbance treatments were attributed according
to a randomized plot design with three replicate blocks of four plots.
The disturbance corresponds to averages of 10 trees removed per hectare
with a diameter at 1.3 m height (DBH) above 50 cm for treatment 1 (T1),
32 trees/ha above 40 cm DBH for treatment 2 (T2) and 40 trees above 40
cm DBH for treatment 3 (T3). Treatments T2 and T3 besides included the
thinning of trees by poison girdling \citep{Schmitt1989, Blanc2009}. The
disturbance intensity was measured as the percentage of aboveground
biomass (\%AGB) lost between the first inventory in 1984 and five years
after disturbance (ref to be found) measured with the BIOMASS R package
\citep{Biomass2018}.

\begin{table}

\caption{\label{tab:Tab1}Intervention table, summary of the disturbance intensity for the 4 plot treatments in Paracou.}
\centering
\begin{tabu} to \linewidth {>{\raggedright}X>{\raggedright}X>{\raggedright}X>{\raggedright}X>{\raggedright}X}
\toprule
Treatment & Timber & Thinning & Fuelwood & \%AGB lost\\
\midrule
Control &  &  &  & 0\\
T1 & DBH $\geq$ 50 cm, commercial species, $\approx$ 10 trees/ha &  &  & $[12\%-33\%]$\\
T2 & DBH $\geq$ 50 cm, commercial species, $\approx$ 10 trees/ha & DBH $\geq$ 40 cm, non-valuable species, $\approx$ 30 trees/ha &  & $[33\%-56\%]$\\
T3 & DBH $\geq$ 50 cm, commercial species, $\approx$ 10 trees/ha & DBH $\geq$ 50 cm, non-valuable species, $\approx$ 15 trees/ha & 40 cm $\leq$ DBH $\leq$ 50 cm, non-valuable species, $\approx$ 15 trees/ha & $[35\%-66\%]$\\
\bottomrule
\end{tabu}
\end{table}

\subsection{Inventories protocol and dataset
collection}\label{protocols}

The study site corresponds to a tropical rainforest with a dominance of
Fabaceae, Chrysobalanaceae, Lecythidaceae and Sapotaceae botanical
families. In the twelve experimental plots of the experiment, all trees
above 10 cm DBH are mapped and measured annually since 1984. Trees are
first identified during inventories with a vernacular name assigned by
the field team, and afterward with a scientific name assigned by a
botanist during regular botanical campaigns. In 1984, specific
vernacular names are given to 62 commercial or common species whereas
more infrequent ones were identified under general identifiers only
distinguiching trees and palm trees. From 2003 botanical campaigns were
conducted every 5 to 6 years to identify all trees at the species level
but identification practices still varied among plots and campaigns.

This variability of protocols raised methodological issues as vernacular
names usually correspond to different botanical species. It resulted in
significant taxonomic uncertainties that had to be propagated to
composition and diversity metrics. The uncertainty propagation was done
through a Bayesian framework reconstituting complete inventories at
genus level from real incomplete ones on the basis of
vernacular/botanical names association. Vernacular names were replaced
through multinomial trials
\(M_v\Big(\big[s_1, s_2, …, s_N\big],\big[\alpha_1, \alpha_2,…, \alpha_3\big]\Big)\)
based on the association probability
\(\big[\alpha_1, \alpha_2,…, \alpha_3\big]\) observed across all
inventories between each vernacular name \emph{v} and the species
\(\big[s_1, s_2, …, s_N\big]\). See appendix 1 and
\citet{Aubry-Kientz2013} for the detailed methodology.

The functional approach used a dataset of 6 functional traits
representing leaf economics (leaves thickness, toughness, total
chlorophyll content and specific leaf area, the leaf area per unit dry
mass) and wood economics (wood specific gravity and bark thickness), and
life history traits (maximum specific height and seed mass). The trait
database came from the BRIDGE project \footnote{http://www.ecofog.gf/Bridge/}
where trait values were assessed from a selection of individuals located
in nine permanent plots in French Guiana, including two in Paracou, and
comprised 294 botanical species pertaining to 157 botanical genera.
Missing trait values were filled using multivariate imputation by
chained equation (mice) from the mice R package \citep{Mice2011}.
Imputations were restricted within genus, or family when samples wre too
scarce, in order to account for the phylogenetic signal of the
functional traits. Whenever a species inventoried was not in the
dataset, it was attributed a set of traits values randomly sampled among
species of the same next higher taxonomic level (same genus or family).
As seed mass information corresponds to a classification into mass
classes, no data filling process was applied and analysis were
restricted to the 414 botanical species recorded.

All composition and diversity metrics corresponded to the average
obtained after 50 iterations of the taxonomic uncertainty propagation
framework and of the filling process of missing trait values.

\subsection{Composition and diversity
metrics}\label{composition-and-diversity-metrics}

To counter taxonomic uncertainties due to the variability of botanical
identification protocols (see {[}\#protocols{]}), the taxonomic
composition and diversity analysis were conducted at the genus level,
\emph{i.e.} referring to the genus of observed or trialed botanical
names. Trajectories of communities taxonomic and functional variations
in composition after disturbance were followed in a two-dimensional
ordination space the 30 years monitored. Two NMDS were conducted to map
either taxonomic flora inventories or communities functional composition
based on the 7 leaf, stem and life history traits (without seed mass
classes). In both cases the NMDS were performed using occurrence-based
(Jaccard) and abundance-based (Bray-Curtis) dissimilarity measures.
Trajectories along time in the plan were reported through the euclidean
distance of successive inventories to the reference inventories in 1989,
5 years after disturbance, when the uncertainty degree did not exceed
30\% of undetermined trees. The trajectories of the leaf and stem and
life traits were also visualized with the community weighted means
(CWM), representing the average trait value in a community weighted by
relative abundance of the species carrying each value
\citep{Diaz2007, Garnier2004}. To compensate the intrinsinc difference
among plots the trajectories corresponded to the differences along time
with the reference inventory in 1989. Species seed mass corresponded to
5 classes of increasing mass, seed mass trajectories were therefore
reported as the proportion of each class in the inventories.

The taxonomic diversity was assessed through Richness and the Hill
number translation of Shannon and Simpson indices \citep{Hill1973}.
These three indices belong to the set of HCDT or generalized entropy,
respectively corresponding to the 0, 1 and 2 order of diversity (q),
which proved well suited for diversity studies
\citep{Patil1982, Tothmeresz1995}. The functional diversity was reported
using the Rao index of quadratic entropy which combines species
abundance distribution and average pairwise dissimilarity based on all
functional and life traits.

The impacts of initial disturbance were first tested with the spearman
rank correlation between the extremum of taxonomic and functional
metrics reached over the 30 years and the initial \%AGB removed. Then
they were analysed through the linear correlations between Simpson and
Rao diversities and the initial \%AGB removed at 10, 20 and 30 years
after disturbance.

The functional redundancy was measured as the overlap among species in
communities' functional space \citep{Carmona2016}. The samples of the
trait database were first mapped in a 2-dimensional plan from a PCA
analysis. Then, multivariate kernel density estimator associated with
individual trees returned species traits probability distribution (TDP).
Species TDP weighted by species abundance were eventually summed for
each community: the functional redundancy was the sum of TDPs overlap,
expressed as the average number of species that could be removed from
without reducing the functional space (see appendix I for a more
comprehensive sheme).

\section{Results}\label{results}

\subsection{Communities Diversity}\label{communities-diversity}

In the inventories from 1989 (5 years after disturbance) to 2015 (31
years after disturbance), 828388 trees and 591 botanical species
pertaining to 223 genus and 64 botanical families were recorded.
Communities taxonomic diversity trajectories were examined through the
Richness, Shannon and Simpson diversities at genus level, in relation to
the 1989 inventories (5 years after disturbance) (See annexe I). For
undisturbed plots the Richness, Shannon and Simpson diversities remained
stable over the 30 years monitored. In disturbed communities the
taxonomic richness increased after low disturbance intensity, reaching a
maximum gain of 14 botanical genera (plot 3 from treatment 2) while it
followed unimodal trajectories after intense disturbance, decreasing for
ten years before recovering pre-disturbance values. In all disturbed
plots the taxonomic evenness (Shannon and Simpson diversities)
increased, following unimodal trajectories with a maximum, reached after
around 20 years, positively correlated to the disturbance treatment
(\(\rho_{spearman}^{Shannon}=0.86\), and
\(\rho_{spearman}^{Simpson}=0.89\)). Return towards initial evenness
values was beginning after 30 years except for two T3 plots (plots 8 and
12) which evenness still increased, suggesting similar but delayed
trajectories \ref{fig:IDHplot}.

Trajectories of communities functional diversity were examined through
the Rao diversity based on the 7 leaf, stem and life history traits (to
the exception of seed mass). The plot 7 from treatment 1 displayed a
constantly outlying diversity and was removed from the graphical
representation for better readability (see appendix for full graphs). In
undisturbed plots the functional diversity remained stable along the 30
years while in disturbed plots it followed unimodal trajectories with a
return towards initial values that strated around 20 years after
disturbance.

The impact of disturbance was examined specifically through the linear
correlation between the intial \%AGB removed and the Simpson and Rao
diversities (diversities of order 2) after 10, 20 and 30 years
\ref{fig:IDHplot}. The correlation with disturbance intensity was weak
for the Simpson diversity (\(R^2<0.25\)) and only valid from 20 years
after disturbance but it was much stronger for the Rao diversity
(\(0.55<R^2<0.75\)) for all the time studied. Slope of linear
correlations, reflecting the impact of disturbance, was the highest 20
years after disturbance.

\begin{figure*}

{\centering \includegraphics[width=1\linewidth]{WholePlotTrajectories_files/figure-latex/IDHplot-1} 

}

\caption{Upper panels, Trajectories of the Simpson taxonomic diversity \textbf{(a)} and Rao functional diversity \textbf{(b)} over 30 years after disturbance, corresponding to the median and 0.025 and 0.975 percentile observed after 50 iteration of the taxonomic uncertainty propagation and the missing trait value filling processes. Initial treatments are represented by solid lines colors with green for control, blue for T1,orange for T2 and red for T3. Lower panels, Relationship between the initial \%AGB removed and the median values of Simpson \textbf{(c)} and Rao \textbf{(d)} diversities at three times after disturbance. Solid lines colors represent the time, 10 years (yellow), 20 years (orange) and 30 years (brown) after disturbance.}\label{fig:IDHplot}
\end{figure*}

\subsection{Communities composition}\label{communities-composition}

\subsubsection{Taxonomic and functional
trajectories}\label{taxonomic-and-functional-trajectories}

The trajectories of taxonomic and functional composition were visualized
in a two dimensional ordination space mapping the successive inventories
according to their flora and corresponding traits. Classifications were
performed using either abundance-based Bray-Curtis (Figure
\ref{fig:NMDSplans}) or incidence-based Jaccard dissimilarity, both
giving similar results only analysis using Bray-Curtis dissimilarity are
discussed.

While both taxonomic and functional composition remained stable in
undistrubed communities, they followed consistent trajectories over time
after disturbance which revealed significant compositional changes.
According to the mapping of functional traits (see appendix I) these
compositional changes corresponded to shifts towards species with more
acquisitive functional strategies, from communities with high average WD
to high average SLA and chlorophyll content. For disturbed communities
the distance of successive inventories to the 1989 reference inventory
followed unimodal trajectories translating cyclic compositional changes
with a recovery of the initial composition (Figure \ref{fig:NMDSplans}).
The maximum dissimilarity with the initial state was positively
correlated to the disturbance treatment for both taxonomic and
functional composition (\(\rho_{spearman}^{taxonomic}=0.91\) and
\(\rho_{spearman}^{functional}=0.96\) respectively) and the time at
maximum was reached around 26 years after disturbance for taxonomic
composition and 22 years for functional composition.

\begin{figure*}

{\centering \includegraphics[width=1\linewidth]{WholePlotTrajectories_files/figure-latex/NMDSplans-1} 

}

\caption{Trajectories of the plots in terms of flora composition (left panels \textbf{(a)} and \textbf{(c)}) and functional composition (right panels \textbf{(b)} and \textbf{(d)}) regarding the 6 leaf and stem functional traits, the maximum allometric height and seed mass class. Plots trajectories are first represented in the two-dimensional space from the NMDS performed for the 30 years after disturbance based on Bray-Curtis dissimilarity measures between successive inventories (Upper panels \textbf{(a)} and \textbf{(b)}). Then the lower panels (\textbf{(c)} and \textbf{(d)}) represent the euclidean distance to initial condition along the 30 sampled years. Line colors represent the disturbance treatment (green for control, blue for T1,orange for T2 and red for T3). The 0.025 and 0.975 percentile correspond to the variance observed for 50 iteration of the taxonomic uncertainty propagation and functional trait filling processes.}\label{fig:NMDSplans}
\end{figure*}

\subsubsection{Traits community weighted means
(CWM)}\label{traits-community-weighted-means-cwm}

The changes observed in plots functional composition went hand to hand
with consistent trajectories of the 8 functional and life history traits
visualized with the trajectories of community weighted means (CWM) of
leaves economics (leaves thickness, chlorophyll content, toughness and
specific area), wood economics (wood specific gravity, bark thickness),
and life history traits (seed mass and maximum adult height) (Figure
\ref{fig:CWM}).

Except for leaf chlorophyll content, which continued to increase for
some T3 and T2 plots 30 years after disturbance, all traits and seed
mass proportions followed unimodal trajectories either stabilizing or
returning towards their initial values. Thirty years after disturbance
the weighted means of communities specific maximum height at adult stage
(\emph{Hmax}), leaf toughness (\emph{L\_toughness}) and wood specific
gravity (\emph{WD}) remained significantly lower than their initial
value (Figure \ref{fig:CWM}). The weighted means of bark thickness
(\emph{Bark\_thick}) similarly remained substantially higher than
initially for all disturbed plots while the specfic leaf area
(\emph{SLA}) had almost recovered its initial value. For all traits the
maximum difference to initial state was correlated to the disturbance
intensity (\(\rho_{spearman}^{L_{thickness}}=0.67\),
\(\rho_{spearman}^{L_{chloro}}=0.45\),
\(\rho_{spearman}^{L_{toughness}}=-0.43\),
\(\rho_{spearman}^{SLA}=0.93\), \(\rho_{spearman}^{WD}=-0.78\),
\(\rho_{spearman}^{Bark-thickness}=0.88\),
\(\rho_{spearman}^{Hmax}=-0.48\)).

\begin{figure*}

{\centering \includegraphics[width=1\linewidth]{WholePlotTrajectories_files/figure-latex/CWM-1} 

}

\caption{Trajectories of the communities weighted means (CWM) over 30 years after disturbance of 4 leaf traits (Leaf thickness, \emph{L\_thickness}, chlorophyll content, \emph{L\_chloro}, toughness, \emph{L\_toughness} and specific area, \emph{SLA}), 2 stem traits (wood specific gravity, \emph{WD}, and bark thickness, \emph{Bark-thick}) and one life trait (Specific maximum height at adult stage, \emph{Hmax}). Trajectories correspond to the median (solid line) and 0.025 and 0.975 percentile (gray envelope) observed after 50 iteration of the taxonomic uncertainty propagation and the missing trait value filling processes. Initial treatments are represented by solid lines colors with green for control, blue for T1,orange for T2 and red for T3.}\label{fig:CWM}
\end{figure*}

\subsubsection{Functional redundancy}\label{functional-redundancy}

Communities functional redundancy was measured as the sum within
communities the species weighted functional overlap based on the 7 leaf,
stem, and maximum height traits (see appendix I for PCA details).
Communities functional redundancy remained stable in control plots but
after disturbance the redundancy trajectories were quite variable (See
appendix I) and apparently independently of the initial disturbance.
Globally after most intense disturbance (plots T2 and T3) communities
redundancy decreased at first place before increasing to edge, recover
or exceed the initial value.

Considering the functional redundancy restricted to the functional space
of the initial inventory, all disturbed plots followed similar
decreasing humped-shaped trajectories (@ref(fig:RedFun\_rest)). The
maximum redundancy loss was positively correlated with the disturbance
intensity (XX spearman to be measured) and the initial value had not
recovered for any disturbed communities.

\begin{figure}

{\centering \includegraphics[width=1\linewidth]{WholePlotTrajectories_files/figure-latex/RedFun_rest-1} 

}

\caption{Trajectories of the functional redundancy within the initial cpùùunities functional space over 30 years after disturbance. Trajectories correspond to the median (solid line) and 0.025 and 0.975 percentile (gray envelope) observed after 50 iteration of the taxonomic uncertainty propagation and the missing trait value filling processes. Initial treatments are represented by solid lines colors with green for control, blue for T1,orange for T2 and red for T3.}(\#fig:RedFun_rest)
\end{figure}

\section{Discussion}\label{discussion}

\subsection{Decoupled recovery of communities taxonomic and functional
characteristics}\label{decoupled-recovery-of-communities-taxonomic-and-functional-characteristics}

Both communities taxonomic and functional diversity and composition
proved resilient, following similar humped-back trajectories with a
return towards initial values. The resilience of communities functional
characteristics, the most direct link between biodiversity and ecosystem
functioning \citep{Diaz2005}, meant a consistent recovery of ecosystem
processes in the long term \citep{Guariguata2001}. The resilience of
communities taxonomic characteristics, in turn, meant the maintenance of
their initial differences in composition and structure. It suggested
that communities response to disturbance were somehow constraint to
converge towards determined compositions
\citep{Hubbell1999, Molino2001, Anderson2007, Baraloto2012a}.

Although both communities taxonomic and functional characteristics
proved resilient and followed similar humped-back trajectories, the
taxonomic recovery systematically lagged behind the corresponding
functional dynamics. Such delay between functional and taxonomic
dynamics has already been observed for grasslands
\citep{Tilman1997, Mouillot2011} and more recently for tropical forests
\citep{Lohbeck2015, Guariguata2001}. According to the ``vegetation
quantity effect'' \citep{Grime1998} functional trajectories rely on the
pool of dominant species, which diversity and evenness were enhanced
after disturbance, and which rapidly restored the functional diversity
and composition. However, communities evenness remained high so
infrequent species still missed to the taxonomic recovery that
mechanically lagged behind, all the more so that unrecovered species
would be functionally redundant and probably undergo competition
processes.

\subsection{A validation of the intermediate disturbance
hypothesis}\label{a-validation-of-the-intermediate-disturbance-hypothesis}

Validating the IDH, communities trajectories confirmed the diversity
increase after disturbance through an enhanced growth of otherwise less
favored species. This was however completely clear regarding the
functional diversity but much more blurry for communities taxonomic
characteristics.

The taxonomic richness was weakly or negatively impacted by intense
disturbance, as observed on several post logging surveys
\citep{Cannon1998, Baraloto2012a}, while it substantially increased
after low intensity disturbance. Disturbance also enhanced taxonomic
evenness, but its intensity was only weakly and lately correlated to
communities eveness (\emph{i.e.} Simpson diversity). Contrastingly,
disturbance intensity consistently predicted the significant increase of
communities functional diversity for 30 years. Disturbance impacts
involve specific turnover within communities, either among
pre-disturbance survivors or among newly recruited trees. As the
composition of old-growth survivors proved to mirror initial communities
\citep{Herault2018}, disturbance would impact trees recruitment through
the enhanced growth and survival of previously infrequent species and
functional characteristics. Consistently, disturbance resulted in
increasing taxonomic dissimilarity compared to pre-disturbance
communities and functional shifts towards resource-acquisitive
strategies (sharp increase in the SLA, leaf thickness and bark thickness
and decrease in wood density, leaf toughness and maximum height)
\citep{Westoby1998, Wright2004, Reich2014}. Disturbance then caused a
reorganization of the typical high dominance structure of hyperdiverse
mature forests after disturbance, benefiting to pioneers and light
demanding species. Likely, the changes in abiotic environment and
competitive pressure favored pioneers which outcompete others in non
limiting resources but are excluded in mature forests by long-lived,
more resistant and shade tolerant species. Therefore, consistently with
communities dynamics after disturbance relied on species functional
strategy and corresponding ability to fill the environmental niches made
available by disturbance. Recruited species then mixed with
pre-disturbance ones, from which they differed, and constituted a
community all the more diversified that the disturbance was intense
\citep{Molino2001}.

\subsection{The functional redundancu and the resilience of
communities}\label{the-functional-redundancu-and-the-resilience-of-communities}

Both the lag between taxonomic and functional recovery and the middling
consistency of the IDH regarding communities taxonomic diversity entail
a central role of functional redundancy. Functional redundancy,
determinant of communities resilience
\citep{Trenbath1999, Elmqvist2003, Diaz2005}, seemed not to have
recovered 30 years after disturbance.

Although disturbance would be mitigated by communities functional
redundancy, we expect the redundancy itself to be reduced or
re-organized within the functional space. Globally, disturbed
communities did not follow consistent trajectories after disturbance and
30 years after disturbance, irrespective of the intensity, disturbed
communities displayed lower, similar or higher functional redundancy.
The redundancy within the functional space of the initial community
however clearly followed humped-back trajectories which decrease was
determined by the disturbance intensity and which return to initial
state was not observed for any disturbed community. Not only did the
functional redundancy and hence the resilience of pre-disturbance
community decrease, but also did the ones of disturbance-specific
communities increase. It means higher chances to have long lasting or
self-maintained compositional changes towards disturbance resistant
species, lianas or epiphytes \citep{Haddad2008, Burslem2000, Martin2013}
and thus highly question forest's resilience \citep{Chazdon2003a}.
Specificaly, this would impair species contingent to undisturbed
forests, threatening their maintenance, and run the risk to loose
cornerstone species and trigger unexpected ecological consequences
\citep{Jones1994, Diaz2005, Gardner2007}.

\section{Conclusions}\label{conclusions}

our study disentangled tropical forests functional and taxonomic
decoupled response to disturbance, with a rapid recovery of communities
functioning but slower and more variable taxonomic dynamics.
Consistently with the IDH functional trajectories were constrained by
environmental pressures favoring pioneers and light-demanding species
after disturbance, but taxonomic dynamics were more stochastic as they
involved infrequent speciesfunctoinally redundant with the dominants and
thus under competitive pressure. While communities functioning rapidly
recovered, taxonomic trajectories, although consistent and converging
towards full recovery, were much slower and entailed persisting
alteration of communities functional redundancy. the trajectories
therefore suggested a potential sustainability of tropical forests in
the face of quite intense disturbance but only when followed by long
recovery periods, longer than 30 years \citep{Gourlet-Fleury2005}. This
study besides highlighted the central role of recruitment processes
underlying the IDH mechanisms and holding the whole community response
to disturbance.

%----------------------------------------------------------------------------------------
%	REFERENCE LIST
%----------------------------------------------------------------------------------------

\bibliographystyle{mee}
\makeatletter
% The filename has .bib extension the must be eliminated
\filename@parse{references.bib}
% parse stores the file name in base. Extension starts at the first dot, so don't use dots in file names.
\bibliography{\filename@base}
\makeatother


%----------------------------------------------------------------------------------------

\end{document}
