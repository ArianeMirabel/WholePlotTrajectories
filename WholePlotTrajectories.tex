%%%%%%%%%%%%%%%%%%%%%%%%%%%%%%%%%%%%%%%%%
% Article EcoFoG
% Version 2.1 (23/10/2017)
%
% adapté de :
% Stylish Article
% LaTeX Template
% Version 1.0 (31/1/13)
%
% This template has been downloaded from:
% http://www.LaTeXTemplates.com
%
% Original author:
% Mathias Legrand (legrand.mathias@gmail.com)
%
% License:
% CC BY-NC-SA 3.0 (http://creativecommons.org/licenses/by-nc-sa/3.0/)
%
%%%%%%%%%%%%%%%%%%%%%%%%%%%%%%%%%%%%%%%%%


%----------------------------------------------------------------------------------------
%	PACKAGES AND OTHER DOCUMENT CONFIGURATIONS
%----------------------------------------------------------------------------------------

\documentclass[fleqn,10pt]{ArtEcoFoG} % Document font size and equations flushed left

\setcounter{tocdepth}{3} % Show only three levels in the table of contents section: sections, subsections and subsubsections


% Pandoc environments
\usepackage{framed}
\usepackage{fancyvrb}
\providecommand{\tightlist}{%
  \setlength{\itemsep}{0pt}\setlength{\parskip}{0pt}}
\newcommand{\VerbBar}{|}
\newcommand{\VERB}{\Verb[commandchars=\\\{\}]}
\DefineVerbatimEnvironment{Highlighting}{Verbatim}{commandchars=\\\{\}, fontsize=\scriptsize} % Code R
\definecolor{shadecolor}{RGB}{248,248,248}
\newenvironment{Shaded}{\begin{snugshade}}{\end{snugshade}}
\newcommand{\KeywordTok}[1]{\textcolor[rgb]{0.13,0.29,0.53}{\textbf{{#1}}}}
\newcommand{\DataTypeTok}[1]{\textcolor[rgb]{0.13,0.29,0.53}{{#1}}}
\newcommand{\DecValTok}[1]{\textcolor[rgb]{0.00,0.00,0.81}{{#1}}}
\newcommand{\BaseNTok}[1]{\textcolor[rgb]{0.00,0.00,0.81}{{#1}}}
\newcommand{\FloatTok}[1]{\textcolor[rgb]{0.00,0.00,0.81}{{#1}}}
\newcommand{\ConstantTok}[1]{\textcolor[rgb]{0.00,0.00,0.00}{{#1}}}
\newcommand{\CharTok}[1]{\textcolor[rgb]{0.31,0.60,0.02}{{#1}}}
\newcommand{\SpecialCharTok}[1]{\textcolor[rgb]{0.00,0.00,0.00}{{#1}}}
\newcommand{\StringTok}[1]{\textcolor[rgb]{0.31,0.60,0.02}{{#1}}}
\newcommand{\VerbatimStringTok}[1]{\textcolor[rgb]{0.31,0.60,0.02}{{#1}}}
\newcommand{\SpecialStringTok}[1]{\textcolor[rgb]{0.31,0.60,0.02}{{#1}}}
\newcommand{\ImportTok}[1]{{#1}}
\newcommand{\CommentTok}[1]{\textcolor[rgb]{0.56,0.35,0.01}{\textit{{#1}}}}
\newcommand{\DocumentationTok}[1]{\textcolor[rgb]{0.56,0.35,0.01}{\textbf{\textit{{#1}}}}}
\newcommand{\AnnotationTok}[1]{\textcolor[rgb]{0.56,0.35,0.01}{\textbf{\textit{{#1}}}}}
\newcommand{\CommentVarTok}[1]{\textcolor[rgb]{0.56,0.35,0.01}{\textbf{\textit{{#1}}}}}
\newcommand{\OtherTok}[1]{\textcolor[rgb]{0.56,0.35,0.01}{{#1}}}
\newcommand{\FunctionTok}[1]{\textcolor[rgb]{0.00,0.00,0.00}{{#1}}}
\newcommand{\VariableTok}[1]{\textcolor[rgb]{0.00,0.00,0.00}{{#1}}}
\newcommand{\ControlFlowTok}[1]{\textcolor[rgb]{0.13,0.29,0.53}{\textbf{{#1}}}}
\newcommand{\OperatorTok}[1]{\textcolor[rgb]{0.81,0.36,0.00}{\textbf{{#1}}}}
\newcommand{\BuiltInTok}[1]{{#1}}
\newcommand{\ExtensionTok}[1]{{#1}}
\newcommand{\PreprocessorTok}[1]{\textcolor[rgb]{0.56,0.35,0.01}{\textit{{#1}}}}
\newcommand{\AttributeTok}[1]{\textcolor[rgb]{0.77,0.63,0.00}{{#1}}}
\newcommand{\RegionMarkerTok}[1]{{#1}}
\newcommand{\InformationTok}[1]{\textcolor[rgb]{0.56,0.35,0.01}{\textbf{\textit{{#1}}}}}
\newcommand{\WarningTok}[1]{\textcolor[rgb]{0.56,0.35,0.01}{\textbf{\textit{{#1}}}}}
\newcommand{\AlertTok}[1]{\textcolor[rgb]{0.94,0.16,0.16}{{#1}}}
\newcommand{\ErrorTok}[1]{\textcolor[rgb]{0.64,0.00,0.00}{\textbf{{#1}}}}
\newcommand{\NormalTok}[1]{{#1}}
\usepackage{longtable,booktabs}
\usepackage{caption}
% These lines are needed to make table captions work with longtable:
\makeatletter
\def\fnum@table{\tablename~\thetable}
\makeatother
% longtable 2 columns
% https://tex.stackexchange.com/questions/161431/how-to-solve-longtable-is-not-in-1-column-mode-error
\makeatletter
\let\oldlt\longtable
\let\endoldlt\endlongtable
\def\longtable{\@ifnextchar[\longtable@i \longtable@ii}
\def\longtable@i[#1]{\begin{figure}[t]
\onecolumn
\begin{minipage}{0.5\textwidth}\scriptsize
\oldlt[#1]
}
\def\longtable@ii{\begin{figure}[t]
\onecolumn
\begin{minipage}{0.5\textwidth}\scriptsize
\oldlt
}
\def\endlongtable{\endoldlt
\end{minipage}
\twocolumn
\end{figure}}
\makeatother

\usepackage{graphicx,grffile}
\makeatletter
\def\maxwidth{\ifdim\Gin@nat@width>\linewidth\linewidth\else\Gin@nat@width\fi}
\def\maxheight{\ifdim\Gin@nat@height>\textheight0.8\textheight\else\Gin@nat@height\fi}
\makeatother
% Scale images if necessary, so that they will not overflow the page
% margins by default, and it is still possible to overwrite the defaults
% using explicit options in \includegraphics[width, height, ...]{}
\setkeys{Gin}{width=\maxwidth,height=\maxheight,keepaspectratio}

% User-adder preamble
\usepackage{textcomp} \DeclareUnicodeCharacter{B0}{\textdegree}
\usepackage{tabu}
\renewenvironment{table}{\begin{table*}}{\end{table*}\ignorespacesafterend}
\hyphenation{bio-di-ver-si-ty sap-lings}

%----------------------------------------------------------------------------------------
%	ARTICLE INFORMATION
%----------------------------------------------------------------------------------------

\JournalInfo{Hal xxx} % Journal information
\Archive{DOI xxxx} % Additional notes (e.g. copyright, DOI, review/research article)

\PaperTitle{Post-Disturbance Tree Community Trajectories in a Neotropical Forest} % Article title

\Authors{
Ariane MIRABEL\textsuperscript{1*}\\ Eric Marcon\textsuperscript{1}\\ Bruno Hérault\textsuperscript{2 3}
} % Authors
\affiliation{
\textsuperscript{1}UMR EcoFoG, AgroParistech, CNRS, Cirad, INRA, Université des Antilles,
Université de Guyane.\\ \hspace{1em} Campus Agronomique, 97310 Kourou, France.\\\textsuperscript{2}Cirad, Univ montpellier, UR Forests \& Societies.\\ \hspace{1em} Montpellier, France.\\\textsuperscript{3}INPHB, Institut National Polytechnique Félix Houphouet-Boigny\\ \hspace{1em} Yamoussoukro, Ivory Coast.
}
\affiliation{*\textbf{Corresponding author}: ariane.mirabel@ecofog.gf, http://www.ecofog.gf/spip.php?article47} % Corresponding author

\Keywords{mot-clés, séparés par des virgules} % Keywords - if you don't want any simply remove all the text between the curly brackets
\newcommand{\keywordname}{Keywords} % Defines the keywords heading name

%----------------------------------------------------------------------------------------
%	ABSTRACT
%----------------------------------------------------------------------------------------

\Abstract{
Résumé de l'article.
}

%----------------------------------------------------------------------------------------

\usepackage{amsthm}
\newtheorem{theorem}{Theorem}[section]
\newtheorem{lemma}{Lemma}[section]
\theoremstyle{definition}
\newtheorem{definition}{Definition}[section]
\newtheorem{corollary}{Corollary}[section]
\newtheorem{proposition}{Proposition}[section]
\theoremstyle{definition}
\newtheorem{example}{Example}[section]
\theoremstyle{definition}
\newtheorem{exercise}{Exercise}[section]
\theoremstyle{remark}
\newtheorem*{remark}{Remark}
\newtheorem*{solution}{Solution}
\begin{document}

\selectlanguage{english}

\flushbottom % Makes all text pages the same height

\maketitle % Print the title and abstract box

\tableofcontents % Print the contents section

\thispagestyle{empty} % Removes page numbering from the first page

%----------------------------------------------------------------------------------------
%	ARTICLE CONTENTS
%----------------------------------------------------------------------------------------






















\section{Introduction}\label{introduction}

The large areas covered with tropical forests worldwide hold crucial
economic, social and cultural value. They provide wood and multiple
non-timber forest products, shelter for a diversified fauna, regulate
the local climate, support the carbon, water and nutrient cycles, and
ensure cultural and human well-being. The simultaneous increase of
forests products demand and substantial climatic changes heightened the
pressure on the remaining forests
\citep{Gibson2011a, Morales-Hidalgo2015} and threatened the maintenance
of communities structure, composition and functioning and their
underlying dynamics \citep{Anderson-Teixeira2013, Sist2015}.

In tropical forests communities are shaped by a constant range of
disturbance that changes the abiotic environment, as the light, the heat
and water fluxes, and the biotic interaction and competitive pressure.
The cornerstone of tropical forests ecology is then to understand their
response to disturbance and its mechanisms
\citep{White2001, Chazdon2003a}. Forests response has been largely
studied through structural parameters rapid and convenient to measure,
as aboveground biomass, tree height or stem density. These structural
parameters have then been sucessfully modeled and allowed to assess the
maintenance of ecosystems processes and services
\citep{Denslow2000, Blanc2009, Rutishauser2016}. However the response of
tree species diversity, which is determinant of ecosystems productivity,
stability and functioning \citep{Tilman2014} and would be most probably
impacted by post-disturbance environmental changes
\citep{CazzollaGatti2014}, remains unclear.

In the short-term diversity dynamics demonstrated negligible or even
positive impacts of disturbance on communities diversity, which have
been formalized by the intermediate disturbance hypothesis (IDH) stating
a maximized species diversity at intermediate disturbance intensity
\citep{Molino2001, Kariuki2006a, Berry2008a}. However, validations of
the IDH in the long term remain scarce and mainly rely on the analysis
of species richness that gives limited or misleading information on
forests recovery and functioning \citep{Martin2015, Chaudhary2016}. More
releveant monitoring would encompass communities composition, that is
crucial for conservation issues, and evenness, that reveals ecological
rules underlying communities' structure
\citep{Magurran1988, Lavorel2002, Bellwood2006}. Furthermore, the
functional approach accounting for species biological attributes would
be insightful as it reveals species fitness and ecosystems functioning
\citep{Violle2007b, Moretti2009, Baraloto2012a, Scheiter2013}. In that
respect major functional traits related to species ecology and
performance were largely adopted through a relevant framework
\citep{Diaz2005, Villeger2008a}. The functional trait-based approach,
for example, highlighted in tropical rainforests the environmental
filters fostering disturbance resistant species with rapid growth and
efficient resources acquisition \citep{Molino2001, Haddad2008}. This
translated by shifts from ``conservative'' slow-growing species dealing
with scarce resources, to ``acquisitive'' fast-growing species with
rapid and efficient use of abundant resources
\citep{TerSteege2001, Reich2014, Herault2011}. It was mirrored by shifts
in key functional traits related to resource acquisition (leaf area,
density and chlorophyll content, and stem specific gravity and bark
thickness), tree growth and reproduction life history traits (seed mass
and maximum height) \citep{Wright2004, Westoby2006a, Chave2009b}. A
proper monitoring of communities response should therefore encompass
taxonomic and functional diversity and composition measures to test the
validity of the IDH in the long term for tropical forests, and clarify
the resilience of communities evenness, composition, and also
functioning. The trajectories followed by all these facets would
highlight the role of deterministic processes, like competitive
exclusion or abiotic selection, and the communities' convergence
maintaining intrinsinc differences in diversity and composition,which is
as much insgights for future adaptive conservation strategies
\citep{Adler2007}.

Here we investigated over 30 years the response of 75 ha of forests
plots set up on a gradient of disturbance intensity, from 10 to 60\% of
ecosystem biomass removed. We made use of a large functional traits
database browsing major leaf, stem and seed traits and species maximum
height to draw the trajectories over time of communities taxonomic and
functional composition and diversity. Specifically, we (i) tested the
validity of the IDH in the long term for tropical hiperdiverse forest
and highlighted the ecological rules shaping their response to
disturbance, (ii) clarified the different facets of communities
resilience in terms of communities composition, diversity and
functioning (iii) questioned the completeness of communities recovery
given the altered functional redundancy.
\textgreater{}\textgreater{}revoir la fin

\section{Material and Methods}\label{material-and-methods}

\subsection{Study site}\label{study-site}

Paracou station in French Guiana (5°18'N and 52°53'W) is located in a
lowland tropical rain forest corresponding to a tropical wet climate
with mean annual precipitation averaging 2980 mm.y\textsuperscript{-1}
(30-y period) with a 3-month dry season (\textless{} 100
mm.month\textsuperscript{-1}) from mid-August to mid-November, and a
one-month dry season in March \citep{Wagner2011}. Elevation ranges
between 5 and 50 m and mean annual temperature is 26°C. Soils correspond
to thin acrisols over a layer of transformed saprolite with low
permeability generating lateral drainage during heavy rains. The
experiment corresponds to a network of twelve 6.25ha plots that have
undergone a gradient of three logging, thinning and fuelwood treatments
(Table \ref{tab:Tab1}). Disturbance treatments were attributed according
to a randomized plot design with three replicate blocks of four plots.
The disturbance corresponds to averages of 10 trees removed per hectare
with a diameter at 1.3 m height (DBH) above 50 cm for treatment 1 (T1),
32 trees/ha above 40 cm DBH for treatment 2 (T2) and 40 trees above 40
cm DBH for treatment 3 (T3). Treatments T2 and T3 besides included the
thinning of trees by poison girdling \citep{Blanc2009}.

\begin{table}

\caption{\label{tab:Tab1}Intervention table, summary of the disturbance intensity for the 4 plot treatments in Paracou.}
\centering
\begin{tabu} to \linewidth {>{\raggedright}X>{\raggedright}X>{\raggedright}X>{\raggedright}X>{\raggedright}X}
\toprule
Treatment & Timber & Thinning & Fuelwood & \%AGB lost\\
\midrule
Control &  &  &  & 0\\
T1 & DBH $\geq$ 50 cm, commercial species, $\approx$ 10 trees/ha &  &  & $[12\%-33\%]$\\
T2 & DBH $\geq$ 50 cm, commercial species, $\approx$ 10 trees/ha & DBH $\geq$ 40 cm, non-valuable species, $\approx$ 30 trees/ha &  & $[33\%-56\%]$\\
T3 & DBH $\geq$ 50 cm, commercial species, $\approx$ 10 trees/ha & DBH $\geq$ 50 cm, non-valuable species, $\approx$ 15 trees/ha & 40 cm $\leq$ DBH $\leq$ 50 cm, non-valuable species, $\approx$ 15 trees/ha & $[35\%-56\%]$\\
\bottomrule
\end{tabu}
\end{table}

\subsection{Inventories protocol and dataset
collection}\label{inventories-protocol-and-dataset-collection}

The study site corresponds to a tropical rainforest with a dominance of
Fabaceae, Chrysobalanaceae, Lecythidaceae and Sapotaceae botanical
families. In the twelve experimental plots of the experiment, all trees
above 10 cm DBH were mapped and measured annually since 1984. During
inventories, trees were first identified with a vernacular name assigned
by the field team, and afterward with a scientific name assigned by a
botanist during regular botanical campaigns. In 1984, specific
vernacular names were given to 62 commercial or common species whereas
other less common species were identified under two identifiers only
separating trees and palm trees. The botanical campaigns carried every 5
to 6 years to identify all trees at the species level only started in
2003 and identification practices varied among plots and successive
campaigns. This raised methodological issues as vernacular names usually
correspond to different botanical species, resulting in significant
taxonomic uncertainties that had to be propagated to composition and
diversity metrics through a Bayesian framework. The uncertainty
propagation was done by the replenishment of inventories completed at
genus level from real incomplete ones on the basis of
vernacular/botanical names association.

Vernacular names were replaced through multinomial trials
\(M_v\Big(\big[s_1, s_2, …, s_N\big],\big[\alpha_1, \alpha_2,…, \alpha_3\big]\Big)\)
based on the observed association probability
\(\big[\alpha_1, \alpha_2,…, \alpha_3\big]\) between each vernacular
name \emph{v} and the species \(\big[s_1, s_2, …, s_N\big]\) recorded in
the inventory. See appendix 1 and \citet{Aubry-Kientz2013} for the
detailed methodology.

The functional diversity metrics used a dataset for 6 functional traits
representing leaf economics (leaves thickness, toughness, total
chlorophyll content and specific leaf area, the leaf area per unit dry
mass) and wood economics (wood specific gravity and bark thickness), and
life history traits (maximum specific height and seed mass).

The trait database came from the BRIDGE project \footnote{http://www.ecofog.gf/Bridge/}
where trait values were assessed from a selection of individuals located
in nine permanent plots in French Guiana, including two in Paracou.
Missing trait values were filled using multivariate imputation by
chained equation (mice) restricted to samples pertaining to the next
higher taxonomic level, in order to account for the phylogenetic signal
of the functional traits. The dataset comprised 294 botanical species
pertaining to 157 botanical genera.

Whenever a species inventoried was not in the dataset, it was attributed
a set of traits values randomly sampled among species of the same next
higher taxonomic level. As seed mass information corresponds to a
classification into mass classes, no data filling process was applied so
analysis were performed considering the 414 botanical species of the
seed mass dataset. All composition and diversity metrics corresponded to
the average obtained after 50 iterations of the taxonomic uncertainty
propagation framework and of the filling process of missing trait
values.

\subsection{Composition and diversity
metrics}\label{composition-and-diversity-metrics}

To counter the remaining taxonomic uncertainty plots taxonomic
composition and diversity were analysed at the genus level, \emph{i.e.}
referring to the genus of observed or trialed botanical names.
Variations in plots taxonomic and functional composition after
disturbance was visualized by their trajectories in a two-dimensional
ordination space over 30 years. Two NMDS were conducted, either from
taxonomic flora inventories or from plots functional composition based
on the 8 leaf, stem and life history traits. Plots trajectories along
time was reported comparatively to the inventories in 1989, 5 years
after disturbance, which corresponded to first inventory with a
sufficient degree of uncertainty (\textless{}30\% of undetermined
trees). The inventories dissimilarity compared to the reference 1989
inventory was reported using occurrence-based (Jaccard) and
abundance-based (Bray-Curtis) similarity measures. The trajectory of
inventories along time was visualized with the euclidean distance in the
two-dimensional ordination space to the 1989 inventory. The trajectories
of the leaf and stem and life traits were also visualized with the
community weighted means (CWM), representing the average trait value in
a community weighted by relative abundance of the species carrying each
value \citep{Diaz2007, Garnier2004}. To compensate the intrinsinc
difference among plots the trajectories corresponded to the differences
along time with the reference inventory in 1984 (5 year after
disturbance). Species seed mass corresponded to 5 classes of increasing
mass, seed mass trajectories were therefore reported as the proportion
of each class in the inventories.

The taxonomic diversity was assessed through Richness and the Hill
number translation of Shannon and Simpson indices \citep{Hill1973}.

These three indices belong to the set of HCDT or generalized entropy,
respectively corresponding to the 0, 1 and 2 order of diversity (q),
which proved well suited for diversity studies
\citep{Patil1982, Tothmeresz1995}. The functional diversity was reported
using the Rao index of quadratic entropy which combines species
abundance distribution and average pairwise dissimilarity based on all
functional and life traits.

\section{Results}\label{results}

\subsection{Disturbance impacts on communities composition and average
functional traits
values}\label{disturbance-impacts-on-communities-composition-and-average-functional-traits-values}

\subsubsection{Composition trajectories}\label{composition-trajectories}

Over time, 828388 trees and 591 botanical species pertaining to 223
genus and 64 botanical families were recorded. Taxonomic and functional
plots trajectories were examined through the ordination in a two
dimensional ordiantion space of successive inventories from 1989 (5
years after disturbance) to 2015 (31 years after disturbance).
Classifications were performed using either abundance-based Bray-Curtis
(Figure \ref{fig:NMDSplans}) or incidence-based Jaccard dissimilarity,
this last having given similar results (data not shown).

Both taxonomic and functional composition of disturbed plots followed
consistent trajectories over time, translating compositional changes
while control plots remained stable. The addition of functional traits
in the two dimensional space revealed a shift of plots composition
towards more acquisitive functional strategies (disturbed plots changed
from high average WD to high average SLA and chlorophyll content). The
dissimilarity of the successive inventories for disturbed plots compared
to their respective reference inventory in 1989 followed unimodal
trajectories (AppendixI, Figure A2). The maximum dissimilarity of plot
strajectories was positively correlated to the disturbance intensity for
taxonomic (\(\rho_{spearman}^{taxonomic}=0.91\)) as well as functional
composition (\(\rho_{spearman}^{taxonomic}=0.96\)). The time at maximum
dissimilarity for taxonomic composition was reached around 26 years
after disturbance while it was 22 years for functional composition. All
trajectories returned towards the initial composition and translated
cyclic compositional changes as visualized in figure (Figure
\ref{fig:NMDSplans}).

\begin{figure}

{\centering \includegraphics[width=1\linewidth]{WholePlotTrajectories_files/figure-latex/NMDSplans-1} 

}

\caption{Trajectories of the plots in terms of \textbf{(a)} flora composition and \textbf{(b)} functional composition regarding the 6 leaf and stem functional traits,the maximum allometric height and seed mass class in the two-dimensional space from the NMDS performed for the 30 years after disturbance. Distance matrix for NMDS were computed from the Bray-curtis dissimilarity between successive inventories. Line colors represent the disturbance treatment (green for control, blue for T1,orange for T2 and red for T3).}\label{fig:NMDSplans}
\end{figure}

\subsubsection{Traits community weighted
means}\label{traits-community-weighted-means}

The changes observed in plots functional composition went hand to hand
with consistent trajectories of the 8 functional and life history traits
(Leaf thickness, chlorophyll content, toughness and specific area,wood
specific gravity and barck thickness and seed mass and maximum adult
height) visaualized with plots community weighted means (CWM) (Figure
\ref{fig:CWM}).

Except for leaf cholorphyll content, which continued to increase for
some T3 and T2 plots 30 years after disturbance, all traits and seed
mass proportions displayed a unimodal trajectories. The weighted means
of communities specific maximum height at adult stage (\emph{Hmax}),
leaf toughness (\emph{L\_toughness}) and wood specific gravity
(\emph{WD}) remained significantly lower than their initial value and
than these of the control plots (Figure \ref{fig:CWM}). The weighted
means of bark thickness (\emph{Bark\_thick}) similarly remained
substantilly higher than initially for all disturbed plots while the
specfic leaf area (\emph{SLA}) had almost recovered its initial value
and this of the undisturbed plots at the end of the experiment. Once
again the maximum difference to initial state was correlated to the
disturbance intensity (\(\rho_{spearman}^{L_thickness}=0.67\),
\(\rho_{spearman}^{L_chloro}=0.45\),
\(\rho_{spearman}^{L_toughness}=-0.43\), \(\rho_{spearman}^{SLA}=0.93\),
\(\rho_{spearman}^{WD}=-0.78\),
\(\rho_{spearman}^{Bark-thickness}=0.88\),
\(\rho_{spearman}^{Hmax}=-0.48\)).

\begin{figure*}

{\centering \includegraphics[width=1\linewidth]{WholePlotTrajectories_files/figure-latex/CWM-1} 

}

\caption{Trajectories of the communities weighted means (CWM) over 30 years after disturbance of 4 leaf traits (Leaf thickness, \emph{L\_thickness}, chlorophyll content, \emph{L\_chloro}, toughness, \emph{L\_toughness} and specific area, \emph{SLA}), 2 stem traits (wood specific gravity, \emph{WD}, and bark thickness, \emph{Bark-thick}) and one life trait (Specific maximum height at adult stage, \emph{Hmax}). Trajectories correspond to the median (solid line) and 0.025 and 0.975 percentile (gray envelope) observed after 50 iteration of the taxonomic uncertainty propagation and the missing trait value filling processes. Initial treatments are represented by solid lines colorswith green for control, blue for T1,orange for T2 and red for T3.}\label{fig:CWM}
\end{figure*}

\subsection{Trajectories of communities diversity after
disturbance}\label{trajectories-of-communities-diversity-after-disturbance}

\subsubsection{Taxonomic diversity}\label{taxonomic-diversity}

Trajectories of Richness, Shannon and Simpson taxonomic diversity were
examined at genus level in relation to the 1989 inventories (5 years
after disturbance) (Figure \ref{fig:DivTaxo}).

\begin{figure*}

{\centering \includegraphics[width=1\linewidth]{WholePlotTrajectories_files/figure-latex/DivTaxo-1} 

}

\caption{Trajectories of the difference to the 1989 inventories (5 years after disturbance) over 30 years after disturbance of plots communities \textbf{(a)} Richness, \textbf{(b)} Shannon and \textbf{(c)} Simpson diversities. Trajectories correspond to the median (solid line) and 0.025 and 0.975 percentile (gray envelope) observed after 50 iteration of the taxonomic uncertainty propagation. Initial treatments are represented by solid lines colors with green for control, blue for T1,orange for T2 and red for T3.}\label{fig:DivTaxo}
\end{figure*}

For undisturbed plots the Richness, Shannon and Simpson diversity
remained stable over the 30 monitored years. After disturbance the
richness increased when the disturbance intensity was low, with a
maximum increase of 14 botanical genuses (plot 3 from treatment 2). When
the disturbance was intense however, plot staxonomic richness followed a
unimodal decrease with a return to initial values after intense
disturbance. The taxonomic evenness (Shannon and Simpson diversities)
significantly increased whenever the disturbance intensity. The evenness
followed a unimodal trajectory with a just beginning return towards
initial values and a maximum reached around 20 years and positively
correlated to the disturbance intensity
(\(\rho_{spearman}^{Shannon}=0.86\), and
\(\rho_{spearman}^{Simpson}=0.89\)). Only two T3 plots, plots 8 and 12,
remained increasing 30 years after disturbance \ref{fig:DivTaxo}),
suggesting a similar be much delayed unimodal trajectory.

\subsubsection{Functional diversity}\label{functional-diversity}

The functional diversity trajectories of plots based on the 8 leaf, stem
and life history traits was examined through the Rao diversity. For all
undisturbed plots the functional diversity remained comparable to the
reference values 5 years after disturbance (1989 inventories). For all
disturbed plots the functional trajectories followed a unimodal
trajectory with a maximum positively correlated to disturbance intensity
(\(\rho_{spearman}=0.73\)) (Figure \ref{fig:DivFun}). The plot 7 from
treatment 1 that displayed a constantly outlying diversity was removed
from the graphical representation for better readability, see appendix
for full graph. Thirty years after disturbance all plots, whenever the
initial disturbance intensity, regained diversity values similar to
their initial avule and to those of control plots.

\begin{figure}

{\centering \includegraphics[width=1\linewidth]{WholePlotTrajectories_files/figure-latex/DivFun-1} 

}

\caption{Trajectories of the Rao functional diversity over 30 years after disturbance. Trajectories correspond to the median (solid line) and 0.025 and 0.975 percentile (gray envelope) observed after 50 iteration of the taxonomic uncertainty propagation. Initial treatments are represented by solid lines colorswith green for control, blue for T1,orange for T2 and red for T3.and the missing trait value filling processes.}\label{fig:DivFun}
\end{figure}

\section{Discussion}\label{discussion}

\subsection{A validation of the intermediate disturbance
hypothesis}\label{a-validation-of-the-intermediate-disturbance-hypothesis}

The monitoring of disturbed forest communities confirmed a limited
negative impact of disturbance on species richness, as observed on
several post logging surveys \citep{Cannon1998, Baraloto2012a}. Thirty
years after disturbance, the genus richness was restored to initial plot
values after high disturbance intensity while it substantially increased
after low disturbance intensity, reaching a gain of almost 12 genuses
for some plots.

Both richness and evenness after disturbance followed asymptotic
trajectories, sharply increasing for 15 years before stabilizing at
higher values than those of control plots. An increase of communities
evenness stems from a higher homogeneity of species distribution. After
disturbance, communities are made of the old, pre-disturbance survivors
and the newly recruited trees: changes in composition and abundance
distribution are to be found in the recruitment processes and in the
pre-disturbance survivors mortality. The composition of old survivors
proved to mirror the initial community \citep{Herault2018}, which leads
to support a species turnover among recruited trees with enhanced growth
and survival of previously infrequent species. Indeed, the increase in
taxonomic diversity was accompanied by an increase of taxonomic
dissimilarity with plots initial state and a functional shift towards
resource-acquisitive strategies (sharp increase in the SLA, leaf
thickness and bark thickness and decrease in wood density, leaf
toughness and maximum height)
\citep{Westoby1998, Wright2004, Reich2014}. Disturbance then causes a
reorganization of the typical high dominance structure of hyperdiverse
mature forests after disturbance, benefiting to pioneers and light
demanding species. Likely, the changes in abiotic environment and
competitive pressure favored pioneers which outcompete others in non
limiting resources but are excluded in mature forests by long-lived,
more resistant and shade tolerant species.

As stated by the IDH, communities dynamics after disturbance then rely
on species functional strategy and corresponding ability to fill the
environmental niches made available by disturbance. Recruited species
then mix with pre-disturbance ones from which they differ and constitue
a community all the more diversified that the disturbance is intense
\citep{Molino2001}.

\subsection{On the recovery of disturbed
communities}\label{on-the-recovery-of-disturbed-communities}

The trajectories of almost all leaf and stem functional traits, life
history traits,and functional diversity followed a hump-back trajectory
which maximum were positively correlated to the disturbance intensity,
and were recovering towards initial values after 30 years that
trajectory. Functoinal traits and communities functional diversity are
the most direct link between biodiversity and ecosystem functioning
\citep{Diaz2005}: these trajectories reflected a consistent recovery in
the long term of ecosystem processes \citep{Guariguata2001}.

Same hump-back trajectories were followed by the composition of plots
species and functoinal composition. This similarly meant a recovery of
the pre-disturbance state and the maintenance of the initial differences
among plots which supports the idea of a convergence of communities
composition \citep{Hubbell1999, Molino2001, Baraloto2012a}. Although
plots followed homogeneous trends these initial differences in
compositon difference, and probably also some abiotic parameters,
strongly determined the maximum and time path of the trajectories
\citep{Anderson2007}•.

\subsection{Functional redundancy of disturbed
ecosystems}\label{functional-redundancy-of-disturbed-ecosystems}

Despite the consistent recovery of initial diversity and composition
there was a time lag between the trajectories followed by taxonomic and
functional characteristics. Taxonomic composition had longer time path
and while communities had recovered their functional characteristics,
their taxonomic composition and evenness remained altered by
disturbance. This delay between functional and taxonomic dynamics was
already observed for grasslands \citep{Tilman1997, Mouillot2011} and
more recently for tropical forests \citep{Lohbeck2015, Guariguata2001}.
The functional diversity of communities rely on the functional strategy
of dominant species, according to the ``vegetation quantity effect''
\citep{Grime1998}. The functional trajectories of communities are then
driven first by the increase of diversity and evenness of dominant
species and then by the recruitment of species resembling the old
pre-disturbance community which reduced the overall diversity. Although
dominant species restoredthe functional pre-disturbance characteristics,
communities composition remained altered and the infrequent species of
initial communities remain missing to achieve plots recovery. According
to the funcitonal traits trajectories, those infrequent species would
match the dominant functional type and would participate to the
functional redundancy. The functional redundancy, thas is the functional
overlap between species that is typical of the huge biodiversity of
tropical forests \citep{Bellwood2006}, therefore was not restored 30
years after disturbance and this is majr to consider as it defines
forests' resilience \citep{Trenbath1999, Elmqvist2003, Diaz2005}.

Besides the long-term alteration of functional redundancy, the
persistent compositional changes which might have favored disturbance
resistant species \citep{Haddad2008}, lianas or epiphytes
\citep{Martin2013} and would probably have come along with environmental
change, like in the soils nutrient cycling and compaction
\citep{Olander2005}. These presistent changes highly question forest's
resilience\citep{Chazdon2003a}. New conditions would not only be longer
lasting but self-maintained as tied to disturbance regime
\citep{Burslem2000}. Specificaly, this would impair species contingent
to undisturbed forests, threatening their maintenance, and run the risk
to loose cornerstone species and trigger unexpected ecological
consequences \citep{Jones1994, Diaz2005, Gardner2007}.

\section{Conclusions}\label{conclusions}

Our study showed the significant impact of disturbance on tropical
forests communities. It validated the consistency of the IDH in the long
term and revealed the contrasting response of taxonomic and functional
characteristics with the decoupling between communities taxonomic
evenness and their functional diversity and th dominant functional
strategies. Communities recovery remained unachieved but consistent
recovery of communities assembly for the lowest disturbance intensity
but questioned it after higher disturbance.

%----------------------------------------------------------------------------------------
%	REFERENCE LIST
%----------------------------------------------------------------------------------------

\bibliographystyle{mee}
\makeatletter
% The filename has .bib extension the must be eliminated
\filename@parse{references.bib}
% parse stores the file name in base. Extension starts at the first dot, so don't use dots in file names.
\bibliography{\filename@base}
\makeatother


%----------------------------------------------------------------------------------------

\end{document}
