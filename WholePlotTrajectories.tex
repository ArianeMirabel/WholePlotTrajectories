%%%%%%%%%%%%%%%%%%%%%%%%%%%%%%%%%%%%%%%%%
% Article EcoFoG
% Version 2.1 (23/10/2017)
%
% adapté de :
% Stylish Article
% LaTeX Template
% Version 1.0 (31/1/13)
%
% This template has been downloaded from:
% http://www.LaTeXTemplates.com
%
% Original author:
% Mathias Legrand (legrand.mathias@gmail.com)
%
% License:
% CC BY-NC-SA 3.0 (http://creativecommons.org/licenses/by-nc-sa/3.0/)
%
%%%%%%%%%%%%%%%%%%%%%%%%%%%%%%%%%%%%%%%%%


%----------------------------------------------------------------------------------------
%	PACKAGES AND OTHER DOCUMENT CONFIGURATIONS
%----------------------------------------------------------------------------------------

\documentclass[fleqn,10pt]{ArtEcoFoG} % Document font size and equations flushed left

\setcounter{tocdepth}{3} % Show only three levels in the table of contents section: sections, subsections and subsubsections


% Pandoc environments
\usepackage{framed}
\usepackage{fancyvrb}
\providecommand{\tightlist}{%
  \setlength{\itemsep}{0pt}\setlength{\parskip}{0pt}}
\newcommand{\VerbBar}{|}
\newcommand{\VERB}{\Verb[commandchars=\\\{\}]}
\DefineVerbatimEnvironment{Highlighting}{Verbatim}{commandchars=\\\{\}, fontsize=\scriptsize} % Code R
\definecolor{shadecolor}{RGB}{248,248,248}
\newenvironment{Shaded}{\begin{snugshade}}{\end{snugshade}}
\newcommand{\KeywordTok}[1]{\textcolor[rgb]{0.13,0.29,0.53}{\textbf{{#1}}}}
\newcommand{\DataTypeTok}[1]{\textcolor[rgb]{0.13,0.29,0.53}{{#1}}}
\newcommand{\DecValTok}[1]{\textcolor[rgb]{0.00,0.00,0.81}{{#1}}}
\newcommand{\BaseNTok}[1]{\textcolor[rgb]{0.00,0.00,0.81}{{#1}}}
\newcommand{\FloatTok}[1]{\textcolor[rgb]{0.00,0.00,0.81}{{#1}}}
\newcommand{\ConstantTok}[1]{\textcolor[rgb]{0.00,0.00,0.00}{{#1}}}
\newcommand{\CharTok}[1]{\textcolor[rgb]{0.31,0.60,0.02}{{#1}}}
\newcommand{\SpecialCharTok}[1]{\textcolor[rgb]{0.00,0.00,0.00}{{#1}}}
\newcommand{\StringTok}[1]{\textcolor[rgb]{0.31,0.60,0.02}{{#1}}}
\newcommand{\VerbatimStringTok}[1]{\textcolor[rgb]{0.31,0.60,0.02}{{#1}}}
\newcommand{\SpecialStringTok}[1]{\textcolor[rgb]{0.31,0.60,0.02}{{#1}}}
\newcommand{\ImportTok}[1]{{#1}}
\newcommand{\CommentTok}[1]{\textcolor[rgb]{0.56,0.35,0.01}{\textit{{#1}}}}
\newcommand{\DocumentationTok}[1]{\textcolor[rgb]{0.56,0.35,0.01}{\textbf{\textit{{#1}}}}}
\newcommand{\AnnotationTok}[1]{\textcolor[rgb]{0.56,0.35,0.01}{\textbf{\textit{{#1}}}}}
\newcommand{\CommentVarTok}[1]{\textcolor[rgb]{0.56,0.35,0.01}{\textbf{\textit{{#1}}}}}
\newcommand{\OtherTok}[1]{\textcolor[rgb]{0.56,0.35,0.01}{{#1}}}
\newcommand{\FunctionTok}[1]{\textcolor[rgb]{0.00,0.00,0.00}{{#1}}}
\newcommand{\VariableTok}[1]{\textcolor[rgb]{0.00,0.00,0.00}{{#1}}}
\newcommand{\ControlFlowTok}[1]{\textcolor[rgb]{0.13,0.29,0.53}{\textbf{{#1}}}}
\newcommand{\OperatorTok}[1]{\textcolor[rgb]{0.81,0.36,0.00}{\textbf{{#1}}}}
\newcommand{\BuiltInTok}[1]{{#1}}
\newcommand{\ExtensionTok}[1]{{#1}}
\newcommand{\PreprocessorTok}[1]{\textcolor[rgb]{0.56,0.35,0.01}{\textit{{#1}}}}
\newcommand{\AttributeTok}[1]{\textcolor[rgb]{0.77,0.63,0.00}{{#1}}}
\newcommand{\RegionMarkerTok}[1]{{#1}}
\newcommand{\InformationTok}[1]{\textcolor[rgb]{0.56,0.35,0.01}{\textbf{\textit{{#1}}}}}
\newcommand{\WarningTok}[1]{\textcolor[rgb]{0.56,0.35,0.01}{\textbf{\textit{{#1}}}}}
\newcommand{\AlertTok}[1]{\textcolor[rgb]{0.94,0.16,0.16}{{#1}}}
\newcommand{\ErrorTok}[1]{\textcolor[rgb]{0.64,0.00,0.00}{\textbf{{#1}}}}
\newcommand{\NormalTok}[1]{{#1}}
\usepackage{longtable,booktabs}
\usepackage{caption}
% These lines are needed to make table captions work with longtable:
\makeatletter
\def\fnum@table{\tablename~\thetable}
\makeatother
% longtable 2 columns
% https://tex.stackexchange.com/questions/161431/how-to-solve-longtable-is-not-in-1-column-mode-error
\makeatletter
\let\oldlt\longtable
\let\endoldlt\endlongtable
\def\longtable{\@ifnextchar[\longtable@i \longtable@ii}
\def\longtable@i[#1]{\begin{figure}[t]
\onecolumn
\begin{minipage}{0.5\textwidth}\scriptsize
\oldlt[#1]
}
\def\longtable@ii{\begin{figure}[t]
\onecolumn
\begin{minipage}{0.5\textwidth}\scriptsize
\oldlt
}
\def\endlongtable{\endoldlt
\end{minipage}
\twocolumn
\end{figure}}
\makeatother

\usepackage{graphicx,grffile}
\makeatletter
\def\maxwidth{\ifdim\Gin@nat@width>\linewidth\linewidth\else\Gin@nat@width\fi}
\def\maxheight{\ifdim\Gin@nat@height>\textheight0.8\textheight\else\Gin@nat@height\fi}
\makeatother
% Scale images if necessary, so that they will not overflow the page
% margins by default, and it is still possible to overwrite the defaults
% using explicit options in \includegraphics[width, height, ...]{}
\setkeys{Gin}{width=\maxwidth,height=\maxheight,keepaspectratio}

% User-adder preamble
\usepackage{textcomp} \DeclareUnicodeCharacter{B0}{\textdegree}
\usepackage{tabu} \usepackage{caption}
\captionsetup{justification = justified}
\renewenvironment{table}{\begin{table*}}{\end{table*}\ignorespacesafterend}
\hyphenation{bio-di-ver-si-ty sap-lings re-or-gan-i-za-tion post-dis-tur-bance dis-tur-bance}

%----------------------------------------------------------------------------------------
%	ARTICLE INFORMATION
%----------------------------------------------------------------------------------------

\JournalInfo{Hal xxx} % Journal information
\Archive{DOI xxxx} % Additional notes (e.g. copyright, DOI, review/research article)

\PaperTitle{Post-Disturbance Tree Community Trajectories in a Neotropical Forest} % Article title

\Authors{
Ariane MIRABEL\textsuperscript{1*}\\ Eric Marcon\textsuperscript{1}\\ Bruno Hérault\textsuperscript{2 3}
} % Authors
\affiliation{
\textsuperscript{1}UMR EcoFoG, AgroParistech, CNRS, Cirad, INRA, Université des Antilles,
Université de Guyane.\\ \hspace{1em} Campus Agronomique, 97310 Kourou, France.\\\textsuperscript{2}Cirad, Univ montpellier, UR Forests \& Societies.\\ \hspace{1em} Montpellier, France.\\\textsuperscript{3}INPHB, Institut National Polytechnique Félix Houphouet-Boigny\\ \hspace{1em} Yamoussoukro, Ivory Coast.
}
\affiliation{*\textbf{Corresponding author}: ariane.mirabel@ecofog.gf, http://www.ecofog.gf/spip.php?article47} % Corresponding author

\Keywords{Taxonomic and Functional Biodiversity, Neotropical Forests, Disturbance Trajectories, Intermediate Disturbance Hypothesis, Long-term Resilience} % Keywords - if you don't want any simply remove all the text between the curly brackets
\newcommand{\keywordname}{Keywords} % Defines the keywords heading name

%----------------------------------------------------------------------------------------
%	ABSTRACT
%----------------------------------------------------------------------------------------

\Abstract{
Understanding the ecological rules underlying the maintenance of
tropical forests biodiversity, structure, functioning and dynamics is
urgent to anticipate their fate in the global change context. The huge
diversity of tropical forests is often assumed to be regularly reshaped
by natural disturbance yielding a diversity peak at intermediate
intensity. This intermediate disturbance hypothesis (IDH), though,
remains debated and the controversy questions the extent of communities'
resilience regarding their taxonomic and functional facets. To
disentangle the ecological processes driving community response to
disturbance, we analysed the tree community trajectories over 30 years
following a disturbance gradient in a neotropical forest. Specifically, we examined community
functional and taxonomic trajectories with regards to diversity,
composition and redundancy. Functional trajectories were drawn based on
7 leaf, stem and life-history traits. We highlighted the cyclic recovery
of community taxonomic and functional composition. While pre-disturbance
taxonomic differences were maintained over time, functional composition
trajectories were quite similar among communities. The IDH did predict
communities taxonomic diversity response while functional diversity was
enhanced whatever the disturbance intensity. Although consistent, the
recovery of community composition, diversity and redundancy remained
unachieved after 30 years. This acknowledged the need of decades-long
cycles with no disturbance to ensure a complete recovery, and questioned
tropical forest community resilience after repeated disturbances.
}

%----------------------------------------------------------------------------------------

\begin{document}

\selectlanguage{english}

\flushbottom % Makes all text pages the same height

\maketitle % Print the title and abstract box

\tableofcontents % Print the contents section

\thispagestyle{empty} % Removes page numbering from the first page

%----------------------------------------------------------------------------------------
%	ARTICLE CONTENTS
%----------------------------------------------------------------------------------------
























\section{Introduction}\label{introduction}

The large areas covered with tropical forests worldwide hold crucial
environmental, economic and social values. They provide wood and
multiple non-timber forest products, shelter a diversified fauna,
regulate the local and regional climates, the carbon, water and nutrient
cycles, and ensure cultural and human well-being. The growing demand in
forests products together with current global changes increases the
pressure on remaining natural forests \citep{Morales-Hidalgo2015} and
threatens the maintenance and dynamics in space and time of communities
structure, composition and functioning \citep{Anderson-Teixeira2013}.

In tropical forests, ecological communities are regularly re-shaped by
natural disturbance events changing both the abiotic environment,
through the fluxes of light, heat and water \citep{Goulamoussene2017},
and the biotic interactions such as competition among species
\citep{Chesson2000}. One of the cornerstone of tropical
forest ecology is to understand the processes and drivers of ecosystems
response to disturbance \citep{Chazdon2003a}. For now, this has been
largely studied through forest structural parameters such as aboveground
biomass, tree height or stem density
\citep{Piponiot2016, Rutishauser2016} that are rapid and convenient to
measure. These structural parameters have been sucessfully modeled,
giving important insights into the recovery of ecosystem processes and
services \citep{Herault2018}. However the response of forests diversity
and composition remains unclear, albeit it determines the productivity,
stability and functioning of ecosystems \citep{Tilman2014, Liang2016}.
In the short-term, moderate disturbance may lead to positive impacts on
communities diversity, an idea formalized by the intermediate
disturbance hypothesis (IDH) stating a maximized species diversity when
disturbance intensity is not too high \citep{Molino2001, Kariuki2006a}.

Validations of the IDH though remain scarce in the long-term and mainly
rely on the analysis of taxonomic richness \citep{Molino2001}. Taxonomic
richness alone, though, gives limited or misleading information on
forests recovery and functioning \citep{Chaudhary2016}. More
ecological-meaningful analysis would couple richness with (i) evenness
that would reveal the changes in the species abundance distribution and thus the underlying ecological processes and (ii) composition
that is crucial for conservation issues
\citep{Lavorel2002, Bellwood2006}. Furthermore, a functional approach
accounting for species biological attributes would directly link
communities diversity, composition and redundancy to ecosystem
functioning and to its environmental constraints
\citep{Violle2007b, Baraloto2012a}. In that respect, the functional
trait-based approach that focus on major traits related to species
ecology and mediate species performance in a given environment was
sucessfully adopted \citep{Diaz2005}. For instance, the functional
approach revealed in tropical rainforests the deterministic processes
entailing, after disturbance, a functional shift from a dominance of
``conservative'' slow-growing species dealing with scarce resources to
``acquisitive'' fast-growing species with rapid and efficient use of
abundant resources \citep{Reich2014, Herault2011}. This shift is
translated into the trajectories of key functional traits related to
resource acquisition (leaf and stem traits) and life-history traits
(seed mass, maximum size)
\citep{Wright2004, TerSteege2006, Westoby2006a, Chave2009b}. Eventually
a complete overview of communities response to disturbance would
encompass the changes in functional redundancy, that quantifies the
amount of shared trait values among species \citep{Carmona2016}. The
high functional redundancy of hyperdiverse tropical forests
\citep{Bellwood2006} mitigates the impacts of species removal on
ecosystem functioning and determines the resilience of communities after
disturbance \citep{Elmqvist2003, Diaz2005}.

In this study, we monitored over 30 years the response of 75 ha of
neotropical forest plots set up on a gradient of disturbance intensity,
from 10 to 60\% of ecosystem above-ground biomass (AGB) loss. We made
use of a large functional traits database encompassing major leaf, stem
and life-history traits in order to draw the taxonomic and functional
trajectories in terms of richness, evenness, composition and redundancy.
Specifically, we (i) elucidated community taxonomic and functional
recovery and the underlying ecological processes, (ii) clarified the
validity of the IDH in the long term for tropical forest and its
translation into different trajectories in time, and (iii) questioned
community recovery time.

\section{Material and Methods}\label{material-and-methods}

\subsection{Study site}\label{study-site}

Paracou station in French Guiana (5\textdegree 18'N and
52\textdegree 53'W) is located in a lowland tropical rain forest in a
tropical wet climate with mean annual temperature of 26\textdegree C,
mean annual precipitation averaging 2980 mm.y\textsuperscript{-1} (30-y
period) and a 3-month dry season (\textless{} 100
mm.month\textsuperscript{-1}) from mid-August to mid-November, and a
one-month dry season in March \citep{Wagner2011}. Elevation ranges
between 5 and 50 m and soils correspond to thin acrisols over a layer of
transformed saprolite with low permeability generating lateral drainage
during heavy rains.

The experiment is a network of twelve 6.25ha plots that underwent a
disturbance gradient of three logging, thinning and fuelwood cutting
treatments (Table \ref{tab:Tab1}) according to a randomized plot design
with three replicate blocks of four plots. The disturbance corresponds
to averages of 10 trees removed per hectare with a diameter at 1.3 m
height (DBH) above 50 cm for treatment 1 (T1), 32 trees/ha above 40 cm
DBH for treatment 2 (T2) and 40 trees/ha above 40 cm DBH for treatment 3
(T3). Treatments T2 and T3 besides included the thinning of trees by
poison girdling \citep{Schmitt1990}. The disturbance intensity was
measured as the percentage of aboveground biomass (\%AGB) lost between
the first inventory in 1984 and five years after disturbance
\citep{Piponiot2016} estimated with the BIOMASS R package
\citep{Biomass2018}.

\begin{table}

\caption{\label{tab:Tab1}Intervention table, summary of the disturbance intensity for the 4 plot treatments in Paracou.}
\centering
\begin{tabu} to \linewidth {>{\raggedright}X>{\raggedright}X>{\raggedright}X>{\raggedright}X>{\raggedright}X}
\toprule
Treatment & Timber & Thinning & Fuelwood & \%AGB lost\\
\midrule
Control &  &  &  & 0\\
T1 & DBH $\geq$ 50 cm, commercial species, $\approx$ 10 trees/ha &  &  & $[12\%-33\%]$\\
T2 & DBH $\geq$ 50 cm, commercial species, $\approx$ 10 trees/ha & DBH $\geq$ 40 cm, non-valuable species, $\approx$ 30 trees/ha &  & $[33\%-56\%]$\\
T3 & DBH $\geq$ 50 cm, commercial species, $\approx$ 10 trees/ha & DBH $\geq$ 50 cm, non-valuable species, $\approx$ 15 trees/ha & 40 cm $\leq$ DBH $\leq$ 50 cm, non-valuable species, $\approx$ 15 trees/ha & $[35\%-56\%]$\\
\bottomrule
\end{tabu}
\end{table}

\subsection{Inventories protocol and dataset
collection}\label{inventories-protocol-and-dataset-collection}

The study site corresponds to a tropical rainforest typical of the
Guiana Shield with a dominance of Fabaceae, Chrysobalanaceae,
Lecythidaceae and Sapotaceae. In the twelve experimental plots, all
trees above 10 cm DBH have been mapped and measured annually since 1984.
Trees are first identified with a vernacular name assigned by the forest
worker team, and afterward with a scientific name assigned by botanists
during regular botanical campaigns. In 1984, specific vernacular names
were given to 62 commercial or common species whereas more infrequent
ones were identified under general identifiers only distinguishing trees
and palm trees. From 2003, botanical campaigns have been conducted every
5 to 6 years to identify all trees at the species level but
identification levels still varied among plots and campaigns.

This variability of protocols in time raised methodological issues as
vernacular names usually correspond to different botanical species. It
resulted in significant taxonomic uncertainties that had to be
propagated to composition and diversity metrics. The uncertainty
propagation was done through a Bayesian framework reconstituting
complete inventories at genus level from real incomplete ones on the
basis of vernacular/botanical names association. Vernacular names were
replaced through multinomial trials based on the association probability
\(\big[\alpha_1, \alpha_2,..., \alpha_V\big]\) observed across all
inventories between each vernacular name \emph{v} and all species
\(\big[s_1, s_2,..., s_N\big]\):

\begin{align}
M_v\Big(\big[s_1, s_2,..., s_N\big],\big[\alpha_1, \alpha_2,..., \alpha_V\big]\Big) \nonumber
\end{align}

See Supplementary Materials -Figure S1 and \citet{Aubry-Kientz2013} for
the detailed methodology.

Six functional traits representing leaf economics (leaves thickness,
toughness, total chlorophyll content and specific leaf area) and stem
economics (wood specific gravity and bark thickness), and life-history
traits (maximum specific height and seed mass) came from the BRIDGE
project \footnote{http://www.ecofog.gf/Bridge/}. Trait values were
assessed from a selection of individuals located in nine permanent plots
in French Guiana, including two in Paracou, and comprised 294 species
pertaining to 157 genera. Missing trait values (10\%) were filled using
multivariate imputation by chained equation \citep{Mice2011}.
Imputations were restricted within genus or family when samples were too
scarce, in order to account for the phylogenetic signal. Whenever a
species was not in the dataset, it was attributed a set of trait values
randomly sampled among species of the next higher taxonomic level (same
genus or family). As seed mass information was classified into classes,
no data filling process was applied and analyses were restricted to the
414 botanical species recorded.

All composition and diversity metrics were obtained after 50 iterations
of the uncertainty propagation framework.

\subsection{Composition and diversity
metrics}\label{composition-and-diversity-metrics}

To counter taxonomic uncertainties due to the variability of botanical
identification levels (in space) and protocols (in time), the taxonomic
composition and diversity analysis were conducted at the genus level.
Taxonomic and functional trajectories of community composition were
followed in a two-dimensional NMDS ordination space. Two NMDS using
abundance-based (Bray-Curtis) dissimilarity measures were conducted to
map either taxonomic or functional composition, the later based on the 7
leaf, stem and life history traits (without seed mass classes).
Trajectories along time were reported through the euclidean distance
between the target inventories and the reference inventories in 1989,
\emph{i.e.} 2 years after disturbance. Univariate trajectories of the
leaf, stem and life-history traits were also visualized with the
community weighted means (CWM) \citep{Diaz2007}. Species seed mass
corresponded to 5 classes of increasing mass, seed mass trajectories
were therefore reported as the proportion of each class in the
inventories (Supplementary materials).

The taxonomic diversity was reported through species richness and
evenness, \emph{i.e} the Hill number translation of the Simpson index
\citep{Hill1973}. These indices belong to the set of HCDT or generalized
entropy, respectively corresponding to the 0 and 2 order of diversity
(q), recomended for diversity studies \citep{Marcon2015b}. The
functional diversity was reported using the functional richness and
functional evenness, \emph{i.e} Rao index of quadratic entropy which
combines species abundance distribution and average pairwise
dissimilarity based on species functional traits.

The impacts of initial disturbance were tested with the spearman rank
correlation between the extrema of taxonomic and functional metrics
reached over the 30 years and the initial \%AGB loss. They were besides
analysed through polynomial regression between (i) taxonomic and
functional richness and evenness and (ii) the initial \%AGB loss at 10,
20 and 30 years after disturbance.

The functional redundancy was measured as the overlap among species in
community functional space \citep{Carmona2016}. The samples of the trait
database were first mapped in a 2-dimensional plan with a PCA analysis.
Then, multivariate kernel density estimator associated with individual
trees returned species traits probability distribution (TDP). Species
TDP weighted by species abundance were eventually summed for each
community. Community functional redundancy was the sum of TDPs overlap,
expressed as the average number of species that could be removed from
without reducing the functional space (Supp. Mat. - Figure S1 for a more
comprehensive sheme).

\section{Results}\label{results}

\subsection{Communities Composition}\label{communities-composition}

From 1989 (2 years after disturbance) to 2015 (28 years after
disturbance), 828 388 individual trees and 591 botanical species
pertaining to 223 genus and 64 families were recorded.

While both taxonomic and functional composition remained stable in
undisturbed communities (Figure \ref{fig:NMDSplans}), they followed
marked and consistent trajectories over post-\break disturbance time. In
disturbed communities, these compositional changes corresponded to
shifts towards species with more acquisitive functional strategies, from
communities with high average WSG to high average SLA and chlorophyll
content (see appendix I). For functional composition, this translated
into cyclic compositional changes with an unachieved recovery of the
initial composition (Figure \ref{fig:NMDSplans}). The maximum
dissimilarity with the initial state was positively correlated to the
disturbance intensity for both taxonomic and functional composition
(\(\rho_{spearman}^{taxonomic}=0.87\) and
\(\rho_{spearman}^{functional}=0.90\) respectively). The maximum value
was reached around 26 years after disturbance for taxonomic composition
and 22 years for functional composition.

\begin{figure*}

{\centering \includegraphics[width=1\linewidth]{WholePlotTrajectories_files/figure-latex/NMDSplans-1} 

}

\caption{Plot trajectories in terms of flora composition (left panels \textbf{(a)} and \textbf{(c)}) and functional composition (right panels \textbf{(b)} and \textbf{(d)}) in a two-dimensional NMDS space. Lower panels (\textbf{(c)} and \textbf{(d)}) represent the euclidean distance to initial condition along the 30 sampled years. Colors are treatments: green (control), yellow (T1), orange (T2), red (T3) with shaded areas the credibility intervals.}\label{fig:NMDSplans}
\end{figure*}

Except for leaf chlorophyll content, which continued to increase for
some T3 and T2 plots 30 years after disturbance, all traits and seed
mass proportions followed unimodal trajectories either stabilizing or
returning towards their initial values.

Maximum height at adult stage (\emph{Hmax}), leaf toughness
(\emph{L\_toughness}) and wood specific gravity (\emph{WSG}) first
decreased and then slightly increased but remained significantly lower
than their initial value (Figure \ref{fig:CWM}). On the other side, Bark
thickness (\emph{Bark\_thick}) and specific leaf area (\emph{SLA})
increased and while \emph{Bark\_thick} remained substantially high after
30 years, \emph{SLA} had almost recovered its initial value. For all
traits, the maximum difference to initial value was correlated to the
disturbance intensity (\(\rho_{spearman}^{L_{thickness}}=0.76\),
\(\rho_{spearman}^{L_{chloro}}=0.60\),
\(\rho_{spearman}^{L_{toughness}}=-0.53\),
\(\rho_{spearman}^{SLA}=0.93\), \(\rho_{spearman}^{WSG}=-0.75\),
\(\rho_{spearman}^{Bark-thickness}=0.71\),
\(\rho_{spearman}^{Hmax}=-0.40\)). The proportions of the three lightest
seed mass classes increased in all disturbed plots, and decreased after
30 years for the lightest class while it stabilized for the two other
(Supp. Mat. - Figure S2).

\begin{figure*}

{\centering \includegraphics[width=1\linewidth]{WholePlotTrajectories_files/figure-latex/CWM-1} 

}

\caption{Trajectories of the communities weighted means (CWM) over 30 years after disturbance of 4 leaf traits (Leaf thickness, \emph{L\_thickness}, chlorophyll content, \emph{L\_chloro}, toughness, \emph{L\_toughness} and specific area, \emph{SLA}), 2 stem traits (wood specific gravity, \emph{WSG}, and bark thickness, \emph{Bark-thick}) and one life history trait (Specific maximum height at adult stage, \emph{Hmax}). Colors are treatments: green (control), yellow (T1), orange (T2), red (T3) with shaded areas the credibility intervals.}\label{fig:CWM}
\end{figure*}

\subsection{Communities richness and
evenness}\label{communities-richness-and-evenness}

For undisturbed plots, taxonomic Richness and Evenness remained stable
over the 30 years monitored. In disturbed communities, after low
disturbance intensity the taxonomic richness increased, reaching a
maximum gain of 14 botanical genera (plot 3 from treatment 2). After
intense disturbance the taxonomic richness followed a more complex
trajectory, decreasing for ten years after disturbance before recovering
to pre-disturbance values. The maximum richness loss or gain after
disturbance was positively correlated to the disturbance intensity
(\(\rho_{spearman}^{Richness}=0.50\)). In all disturbed plots the
taxonomic evenness first increased until a maximum reached after around
20 years. This maximum was positively correlated to the disturbance
intensity (\(\rho_{spearman}^{Evenness}=0.77\)). The evenness then
stabilized except for two T3 plots (plots 8 and 12) for which evenness
kept increasing.

\begin{figure*}

{\centering \includegraphics[width=1\linewidth]{WholePlotTrajectories_files/figure-latex/DivTaxo-1} 

}

\caption{Trajectories over 30 years of the difference with the 1989 inventory (2 years after disturbance) of community taxonomic \textbf{(a)} richness, \textbf{(b)}, taxonomic evenness, \textbf{(c)} functional richness, and \textbf{(d)} functional evenness. Colors are treatments: green (control), yellow (T1), orange (T2), red (T3) with shaded areas the credibility intervals }\label{fig:DivTaxo}
\end{figure*}

The plot 7 from treatment 1 displayed constantly outlying functional
richness and evenness and was removed from the graphical representation
for better readability. In undisturbed plots both functional richness
and evenness remained stable along the 30 years. In disturbed plots,
functional richness and evenness trajectories depended on the
disturbance intensity with their maximum positively correlated to \%AGB
loss \(\rho_{spearman}^{Richness}=0.76\) and
\(\rho_{spearman}^{Evenness}=0.60\). Functional richness and evenness
displayed for low disturbance intensity a low but long-lasting increase
up to a maximum reached after 20-25 years, and for high intensity, a
fast but short increase followed after 10 years by a slow decrease
towards the inital values.

The second-degree polynomial regressions between (i) the \%AGB loss and
(ii) taxonomic and functional richness and evenness after 10, 20 and 30
years best predicted the hump-shaped curve of the disturbance impact
along the disturbance intensity gradient \ref{fig:IDHplot}. The
relationship between the disturbance impact and its intensity was more
markedly hump-shaped for the taxonomic richness than for the taxonomic
evenness. For both functional richness and evenness the relationship was
almost linear. The regression model better predicted the functional
richness and evenness (\(0.55<R^2_{Functional Richness}<0.72\), and
\(0.60<R^2_{Functional Evenness}<0.81\)) than the taxonomic richness and
evenness (\(0.21<R^2_{Taxonomic Richness}<0.4\), and
\(-0.15<R^2_{Taxonomic Evenness}<0.43\) respectively)

\begin{figure*}

{\centering \includegraphics[width=1\linewidth]{WholePlotTrajectories_files/figure-latex/IDHplot-1} 

}

\caption{Relationship between the initial \%AGB loss and community taxonmic richness \textbf{(a)}, taxonomic evenness \textbf{(b)}, functional richness \textbf{(c)},and functional evenness \textbf{(d)} at 10, 20 and 30 years after disturbance}\label{fig:IDHplot}
\end{figure*}

\subsection{Functional redundancy}\label{functional-redundancy}

All disturbed plots had lower functional redundancy than control plots
and followed similar hump-shaped trajectories (\ref{fig:RedFunRest}).
The maximum redundancy loss was positively correlated with the
disturbance intensity (\(\rho_{spearman}=0.47\)) and the initial value
had not recovered for any disturbed communities after 30 years.

\begin{figure}

{\centering \includegraphics[width=1\linewidth]{WholePlotTrajectories_files/figure-latex/RedFunRest-1} 

}

\caption{Trajectories of the functional redundancy within the initial functional space over 30 years after disturbance. Colors are disturbance treatments: green (control), yellow (T1), orange (T2), red (T3) with shaded areas the credibility intervals.}\label{fig:RedFunRest}
\end{figure}

\section{Discussion}\label{discussion}

\subsection{A cyclic recovery of community
composition}\label{a-cyclic-recovery-of-community-composition}

Communities taxonomic and functional composition appeared resilient,
following similar hump-shaped trajectories starting to return towards
pre-disturbance composition after 30 years.

The taxonomic differences among local communities, marked before disturbance
by the distinct starting points on the NMDS axis 2, were maintained
throughout recovery trajectories. More than commonly thought,
post-disturbance trajectories depended on community initial composition,
that partly determined the pool of recruited species and constrained the
trajectories towards the initial composition. The high resilience of
communities taxonomy revealed that species not belonging to the
pre-disturbance community were hardly recruited because of the
commonness of dispersal limitation among tropical tree species
\citep{Svenning2005}.

Conversely, disturbed communities followed functional trajectories
that are highly similar in terms of functional composition. As pre-disturbance surviving
trees mirror the initial community \citep{Herault2018}, changes in
functional composition relied upon the recruitment of species or
functional types that were infrequent or absent before disturbance.
Competitive pioneers became dominant in filling the environmental niches
of high availability of light, space and nutrients vacated by the
disturbance. The recruitment of pioneers changed community functional
composition in the same way for all disturbance intensity towards more
resource-acquisitive strategies, moving community functional composition
right along the first axis in Figure \ref{fig:NMDSplans}
\citep{Westoby1998, Wright2004, Reich2014}. Thereafter long-lived, more
resistant and shade-tolerant species excluded the first established
pioneers and started the recovery of pre-disturbance functional
composition, moving similarly community functional composition left
along the first axis and upward along the second axis in Figure
\ref{fig:NMDSplans}.

These trajectories provided empirical support to the hypothesis that
community assembly is both deterministic and historically convergent at
different levels of community organization. Deterministic, trait-based
processes drove community convergence in functional composition, while
at the same time dispersal limitation maintained their divergence in
taxonomic composition \citep{Fukami2005}.

\subsection{Another perspective on the intermediate disturbance
hypothesis}\label{another-perspective-on-the-intermediate-disturbance-hypothesis}

The IDH well predicted the disturbance impact on community taxonomic
richness, enhanced until an intensity threshold (20-25\% AGB loss), and
to some extent on taxonomic evenness, somewhat decoupled from the
disturbance intensity as already observed in the Guiana Shield
\citep{Baraloto2012a} and in Bornean tropical forests
\citep{Cannon1998}. The disturbance intensity determined the balance in
the community between pre-disturbance surviving trees and those
recruited afterward. The pool of true pioneer species specifically
recruited after disturbance is restricted in the Guiana Shield to a few
common genera (e.g.~Cecropia spp., Vismia spp.) \citep{Guitet2018}.
Below the intensity threshold the size of the surviving community
maintained the pre-disturbance high taxonomic richness while the
recruitment of pioneers, infrequent or absent before disturbance,
increased both community taxonomic richness and evenness. Beyond the
intensity threshold, the disturbance decreased the taxonomic richness of
surviving trees which was not offset by the enrichment of pioneers, so
that the overall community taxonomic richness decreased according to the
disturbance intensity \citep{Molino2001}. For community taxonomic
evenness the disturbance impact was similar but slighter, as the
evenness is less sensitive to the loss of rare species. Taxonomic
evenness rather represented the increasing dominance of pionneers that
balanced the usual hyper-dominance of a few species in tropical forests below the
intensity threshold, thus increasing community overall evenness up to
the intensity threshold beyond which pioneers became in turn highly
dominant and decreased the overall evenness \citep{Baraloto2012a}.

Conversely the IDH was disproved regarding the disturbance impact on
community functional richness and evenness. Irrespective of the
disturbance intensity the recruitment of pionneers, functionally highly
different from the composition of pre-disturbance community, increased
both community functional richness and eveness.

Along time, taxonomic richness trajectories of all disturbed communities
first dropped similarly, following the species loss due to disturbance,
and then displayed a species gain depending on the disturbance
intensity. Up to an intensity threshold, the species gain was all the
more significant that the disturbance intensity increased, with the
establishment of long-lived pioneers enhancing community taxonomic
richness and evenness in the long term. These long-lived pionneers,
functionally quite different from the functional composition, entailed
as well a progressive and long-lasting increase of the functional
richness and evenness \citep{Denslow1980, Molino2001}. Beyond an
intensity threshold, though, a few short-lived pioneers occupied the
vacated environmental space and prevented the establishment of other
species. These short-lived pioneers were functionally very different
from the pre-disturbance community and entailed a rapid and significant
increase of functional richness en evenness. Already after 10 years,
though, short-lived pioneers started to decline and the functional
richness and evenness decreased. Likely this decrease will be followed
by the establishment of long-lasting pioneers, and by the time they
recruit we expect the taxonomic and functional trajectories to catch up
with those observed after intermediate disturbance \citep{Walker2009}.

\subsection{The functional redundancy, key of community
resilience}\label{the-functional-redundancy-key-of-community-resilience}

For 15 years the species loss during disturbance, determined by the
disturbance intensity, commensurately decreased the functional
redundancy within the pre-disturbance functional space. The redundancy
decrease was not compensated in the first place as the first recruited
pionneers were functionally different from the pre-disturbance
functional composition. Progressively though, first established species
were replaced by more competitive long-lived pionneers or
late-successional species resembling more the pre-disturbance functional
composition and restoring the functional redundancy. This replacement
was stochastic and followed the lottery recruitment rules, implying a
recruitment eased for the first recruited species but then increasingly
hampered by the emergence of interspecific competition
\citep{Busing2002}. Along time the recovery of infrequent species was
increasingly difficult, so that the time for the full recovery of the
functional redundancy, in some communities just initiated after 30
years, was extremely difficult to estimate \citep{Elmqvist2003, Diaz2005}.

The long-term impact of disturbance on community functional redundancy
meant a lower resilience of the pre-disturbance communities, with higher
chances to see the persistence of disturbance-specific species at the
expense of late-successional ones \citep{Haddad2008}. Besides, the
long-term recovery of infrequent species increases the risks to loose
cornerstone species, with unexpected ecological consequences
\citep{Jones1994, Chazdon2003a, Diaz2005}. Apart from the functional
characteristics considered here, infrequent species might indeed have
unique functions in the ecosystem or be a key for some fauna
\citep{Schleuning2016}.

\section{Conclusions}\label{conclusions}

Our study revealed community recovery through the combination of
deterministic processes driving their convergence in functional
composition, and dispersal limitation maintaining their divergence in
taxonomic composition. The IDH was validated for community taxonomic
richness and, to some extent, taxonomic evenness but disproved regarding
community functional richness and evennes that were enhanced for any
disturbance intensity by the high functional differences of pioneers
compared to late-successional functoinal composition. The IDH was
translated in time by the recruitment, beyond an intensity threshold, of
short-lived pionneers that prevented in the first times the
establishment of more diverse long-lived pionneers, recruited otherwise
below the intensity threshold. The resilience of tropical forests proved
consistent although several decades-long. Still, the disturbance impact
on communities redundancy cautioned against the risks of infrequent
species loss and the persistence of disturbance-specific communities
\citep{Herault2018}.

%----------------------------------------------------------------------------------------
%	REFERENCE LIST
%----------------------------------------------------------------------------------------

\bibliographystyle{mee}
\makeatletter
% The filename has .bib extension the must be eliminated
\filename@parse{references.bib}
% parse stores the file name in base. Extension starts at the first dot, so don't use dots in file names.
\bibliography{\filename@base}
\makeatother


%----------------------------------------------------------------------------------------

\end{document}
